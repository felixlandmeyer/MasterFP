\section{Diskussion}

Mit einem Germaniumdetektor lassen sich Gammaspektren mit einer extrem genauen Energieauflösung messen.
Die fehlende Untergrundmessung beeinflusst die Auswertung der Spektren nicht merklich, da meist nur Photopeaks betrachtet wurden, die sich sehr stark vom Untergrund abheben und das Untergrundsignal mit steigender Energie schnell abfällt.

Aufgrund der genauen Energieauflösung lassen sich die Kanäle ohne signifikante Unsicherheiten den Photonenenergien zuordnen.
Die Berechnung der Vollenergienachweiswahrscheinlichkeit liefert für die einzelnen Photopeaks kleine Unsicherheiten, jedoch sind die statistischen Unsicherheiten der Ausgleichsrechnung in Gleichung \eqref{eqn:Eff} relativ hoch.

Die Untersuchung der \ce{^137Cs}-Quelle liefert eine gute Einsicht in die Energieauflösung des Germaniumdetektors.
Der schmale Photopeak folgt exakt einer Gausverteilung und liegt nur $\SI{90}{\electronvolt}$ neben dem Literaturwert.
Comptonkante und Rückstreupeak sind, trotz des verrauschten Comptonkontiuums, gut zu identifizieren.
Die hohen Abweichungen bei Betrachtung des Verhältnis der Abschwächungskoeffizienten für Compton- und Photoeffekt im Vergleich mit dem tatsächlichen Verhältnis der detektierten Compton- und Photopeakinhalte sind zu groß, um sie alleine durch Mehrfachstreuung erklären zu können. Bei Energien oberhalb der Comptonkante sind zu wenig Einträge, um diese These zu stärken.

Die gemessene Aktivität der \ce{^133Ba}-Quelle liegt im Bereich der berechneten Aktivität der Probe.

Aus der Untersuchung des unbekannten Strahlers geht hervor, dass dieser Uran 238 enthält. Es wurden zwar noch viele weitere Photopeaks des Spektrums untersucht, konnten aber nicht zweifelsfrei einer Zwerfallskette, oder einzelnen Isotopen zugeordnet werden.
