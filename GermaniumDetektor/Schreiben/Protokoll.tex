\documentclass[captions=tableheading,
  bibliography=totoc,
  titlepage=firstiscover
  ]{scrartcl}
\usepackage{scrhack}
\usepackage[a4paper,top=2.5cm,left=2.5cm,right=2cm,bottom=3cm,bindingoffset=5mm]{geometry}
\usepackage[aux]{rerunfilecheck}
\usepackage{polyglossia}
\setmainlanguage{german}
\usepackage{amsmath}
\usepackage{amssymb}
\usepackage{xfrac}
\usepackage{mathtools}
\usepackage{fontspec}
\usepackage[natbib=true,style=numeric,sorting=none]{biblatex}
\usepackage{stmaryrd}
\addbibresource{lit.bib}
\usepackage[
  math-style=ISO,
  bold-style=ISO,
  sans-style=italic,
  nabla=upright,
  partial=upright,
]{unicode-math}
\setlength{\parindent}{0cm}
\usepackage[
  locale=DE,
  separate-uncertainty=true,
  per-mode=symbol-or-fraction,
]{siunitx}

\usepackage[section, below]{placeins}
\usepackage[
  labelfont=bf,        % Tabelle x: Abbildung y: ist jetzt fett
  font=small,          % Schrift etwas kleiner als Dokument
%  width=0.9\textwidth, % maximale Breite einer Caption schmaler
  format=plain,
  indention=1em, % Abbildung sticht links etwas hervor
]{caption}
\usepackage{graphicx}
\usepackage{xcolor}
\usepackage{grffile}
\usepackage{subcaption}
\usepackage{booktabs}
\usepackage{float}
\floatplacement{figure}{htbp}
\floatplacement{table}{htbp}
\usepackage[unicode]{hyperref}
\usepackage{bookmark}
\usepackage{microtype}
\usepackage{pdfpages}
\usepackage{mhchem}
\usepackage{dsfont}
\begin{document}
\begin{titlepage}
  \centering
  %\includegraphics[width=0.5\textwidth]{../Logo.pdf}\par\vspace{2cm}
  {\scshape\LARGE Technische Universität Dortmund \par}
  \vspace{1.5cm}
  {\scshape\Large Fortgeschrittenen Praktikum \par}
  \vspace{1.5cm}
  {\scshape\LARGE\bfseries Germanium Detektor\par}
  \vspace{2.5cm}
  {\Large Melina Helfrich, \texorpdfstring{\href{mailto:melina.helfrich@tu-dortmund.de}{melina.helfrich@tu-dortmund.de}}\par\par
  \par\vspace{0.25cm}
  Felix Landmeyer, \texorpdfstring{\href{mailto:felix.landmeyer@tu-dortmund.de}{felix.landmeyer@tu-dortmund.de}}\par\par}
  \vspace{2cm}
  {\scshape Durchführung am 04.11.2019 \par
    Abgabe am ..2019 \par
    %Korrekturabgabe am ..2019
    }
\end{titlepage}
\newpage
\tableofcontents
%\newpage
\setcounter{page}{3}
%\section{Einleitung}

Ziel dieses Versuchs ist der Aufbau und die Kontrastmessung eines Sagnac-Interferometers, sowie die Bestimmung der Brechungsindices von Luft in Abhängigkeit des Luftdrucks und Glas mit eben diesem Interferometer.

\section{Theorie}

Im Folgenden soll zuerst die Entstehung von Interferenzeffekten erläutert werden \cite{Lit1}.
Die Wellencharakteristik von elektromagnetischer Strahlung kann durch
\begin{equation*}
  \vec{E}(\vec{r},t) = \vec{E_0} \cdot e^{i(\omega t - \vec{k}\vec{r})}
\end{equation*}
beschrieben werden.
Dabei ist $\vec{k}$ der Wellenvektor der Strahlung, $\omega$ ihre Frequenz und $\vec{E_0}$ die Amplitude.
Bei kohärentem Licht und gleicher Polarisationsrichtung kommt es bei Überlagerung zur Interferenz.
Für zwei interferierende Wellen
\begin{align*}
  \vec{E_1}(\vec{r},t) &= \vec{E_{1,0}} \cdot e^{i(\omega t - \vec{k}\vec{r})} \, , \\
  \vec{E_2}(\vec{r},t) &= \vec{E_{2,0}} \cdot e^{i(\omega t - \vec{k}\vec{r} + \gamma)} \, ,
\end{align*}
wobei $\vec{E_2}$ eine Phasenverschiebung $\gamma$ zu $\vec{E_1}$ aufweist, folgt für die Intensität
\begin{equation}
  I = \Bigl \langle |\vec{E_1} - \vec{E_2}|^2 \Bigr \rangle = E_{1,0} + E_{2,0} + 2 E_{1,0} E_{2,0} \cos(\gamma) \, .
  \label{eqn:Intensität}
\end{equation}
Für gleiche Amplituden $E_{1,0} = E_{2,0} = E$ folgt für die Intensität
\begin{equation}  \label{eqn:Gl2}
  \begin{aligned}
  \gamma &= \pi &\rightarrow& &I_\text{min.} &= 0 \, , \\
  \gamma &= 0 &\rightarrow& &I_\text{max.} &= 4E \, .
\end{aligned}
\end{equation}

\bigskip

Bislang wurden nur Wellen mit gleicher Polarisation und Polarisationsrichtung betrachtet.

Das Herzstück des Sagnac-Interferometers bildet jedoch ein \textit{Polarizing Beam Splitter Cube} (PBSC).
\begin{wrapfigure}[15]{r}{.4\textwidth}
\includegraphics[width = .3\textwidth]{bilder/sketch-polarizing-cube-beamsplitter.png}
\caption{Schematische Zeichnung eines PBSC und die Polarisationsrichtungen des Lichts \cite{PBSC}.}
\label{fig:PBSC}
\end{wrapfigure}
Dieser transmittiert Licht, welches parallel zur horizontalen Achse des PBSC polarisiert ist und reflektiert Licht, welches vertikal polarisiert ist (vlg. Abbildung \ref{fig:PBSC}).
Daher muss für die weitere Betrachtung die Polarisationsrichtung des, auf den PBSC treffenden, Lichtes beachtet werden.
Linear polarisiertes Licht $E$, dessen Polarisationsrichtung den Winkel $\phi$ zur horizontalen Achse des PBSC aufweist, wird durch diesen in den transmittieren Anteil
\begin{equation*}
  E_{\_} = E \cos(\phi)
\end{equation*}
und den reflektierten Anteil
\begin{equation*}
  E_{\bot} = E \sin(\phi)
\end{equation*}
aufgeteilt.
\medskip
Bei Interferenz dieser beiden Strahlen ergibt sich nach Gleichung \eqref{eqn:Intensität} für die Intensität der Interferenz
\begin{equation*}
  I = E^2 \cdot (1 + 2 \cos(\phi) \sin(\phi) \cos(\gamma))
\end{equation*}
aus dieser folgt für die Phasenverschiebung analog zu \eqref{eqn:Gl2}
\begin{equation}  \label{eqn:Gl3}
  \begin{aligned}
  \gamma &= \pi &\rightarrow& &I_\text{min.} &= E^2 \cdot (1 − \sin(2\phi) \, , \\
  \gamma &= 0 &\rightarrow& &I_\text{max.} &= E^2 \cdot (1 + \sin(2\phi) \, .
\end{aligned}
\end{equation}

Eine wichtige Kenngröße eines Sagnac Interferometers ist sein Kontrast
\begin{equation}
  K = \frac{I_\text{max.} - I_\text{min.}}{I_\text{max.} + I_\text{min.}} = \sin(2\phi) \, .
  \label{eqn:Kontrast}
\end{equation}
Es wird schnell ersichtlich, dass dieser bei einer Polarisationsrichtung von $\phi = \SI{45}{\degree}$ maximal ist, da $\cos(\SI{45}{\degree}) = \sin(\SI{45}{\degree})$ und damit der reflektierte Anteil des eingestrahlten Lichtes genauso groß ist, wie der transmitierte Anteil.


\subsection{Brechungsindex von Glas}
Fällt Licht im Winkel $\delta$ auf eine planparallele Glasplatte, beträgt die Phasenverschiebung $\Delta \gamma$ nach \cite{anleitung} für kleine Winkel $\delta$
\begin{equation} \label{eqn:Phasendifferenz}
  \Delta \gamma = \frac{2 \pi}{\lambda_\text{vac.}} \, T \, \frac{n-1}{2n} \delta^2 \, .
\end{equation}
Die Phasenverschiebung ist somit abhängig von der Wellenlänge $\lambda_\text{vac.}$ des Lichts im Vakuum, der Dicke $T$ der Glasplatte und dem Brechungsindex $n$ des Glases.
Durch Veränderung des Winkels $\delta$ können also Phasendifferenzen erzeugt werden, die zu Interferenzeffekten mit Licht führen, welches keine Glasplatte passiert hat.
Aus der Anzahl $M = \sfrac{\Delta \gamma}{2 \pi}$ der Phasenverschiebungen um \SI{360}{\degree} bei Drehung der Glasplatte um den Winkel $\delta$ kann nach Umstellen der Gleichung \eqref{eqn:Phasendifferenz} der Brechungsindex
\begin{equation}\label{eqn:n_Glas}
  n = \left( 1 - \frac{2 M \lambda_\text{vac.}}{T \delta^2} \right)^{-1}
\end{equation}
berechnet werden.


\subsection{Brechungsindex von Gas}

Wird der Lichtstrahl durch einen mit Gas gefüllten Raum der Länge $L$ geleitet, folgt
\begin{equation} \label{eqn:n_Gas}
  \begin{split}
  \Delta \gamma &= \frac{2 \pi}{\lambda_\text{vac.}} \, (n-1) \, L \\
  \leftrightarrow n &= \frac{\lambda_\text{vac.}}{L} \,  M + 1
  \end{split}
\end{equation}
Es ist also möglich aus einer gemessenen Anzahl $M$ an $2 \pi$ Phasenverschiebungen den Brechungsindex $n$ des Gases zu bestimmen.
Weiterhin kann aus einer solchen Messung in Abhängigkeit des Drucks $p$ durch
\begin{equation*}
  n = \frac{\Delta n}{\Delta p} \, p + n_{p=0}
\end{equation*}
der Brechungsindex eines Gases für beliebige Drücke abgeschätzt werden.

%\newpage
\section{Auswertung}
Der genutze Laser hat eine Wellenlänge von \SI{623.99}{\nano\meter}.

\subsection{Kontrast des Interferometers}
Um den maximalen Kontrast des Interferometers zu bestimmen, wird für jeden Polarisationswinkel $\phi$  der Kontrast aus den Diodenspannungen der Intensitätsmessung mit Formel \ref{eqn:Kontrast} berechnet und in Tabelle \ref{tab:Kontrast} aufgelistet.
Weiterhin werden in Abbildung \ref{fig:Kontrast} die berechneten Kontraste gegen den Polarisationswinkel aufgetragen und eine Ausgleichsrechnung der Form

\begin{equation*}
  K = A \cdot |\sin(2 \phi)|
\end{equation*}

durchgeführt. Als maximaler Kontrast folgt
\begin{equation}
  K_\text{max} = A = \num{0.71(5)} \, .
  \label{eqn:Kontrast}
\end{equation}

Es ist deutlich zu sehen, dass der maximale Kontrast deutlich kleiner als eins ist. Für die Durchfuhrung des Versuches wurde der Polarisationswinkel auf \SI{45}{degree} gestellt.

\begin{figure}[H]
  \centering
  \includegraphics[width = .5\textwidth]{Auswertung/Plots/Kontrast.png}
  \caption{Kontrast des Interferometers in Abhängigkeit des Polarisationswinkels. }
  \label{fig:Kontrast}
\end{figure}

\begin{table}[H]
  \centering
  \caption{Spannungen $U_\text{min}$, $U_\text{max}$ und Kontrast $K$ des Interferometers in Abhängigkeit des Polarisationswinkels $\phi$. Für die Versuchsdurchführung wurde der Polarisationswinkel auf \SI{45}{\degree} gestellt.}
  \label{tab:Kontrast}
  \begin{tabular}{cccc}
    \toprule
    $\phi \, / \,  \si{\degree}$ & $U_\text{max} \, / \, \si{\volt}$ &  $U_\text{min} \, / \, \si{\volt}$ & Kontast \\
    \midrule
    0   & \num{1.060} &  \num{0.880} & \num{0.09} \\
    15  & \num{0.634} &  \num{0.479} & \num{0.14} \\
    30  & \num{0.560} &  \num{0.243} & \num{0.24} \\
    40  & \num{0.692} &  \num{0.090} & \num{0.76} \\
    45  & \num{0.855} &  \num{0.095} & \num{0.80} \\
    50  & \num{0.967} &  \num{0.136} & \num{0.70} \\
    60  & \num{0.972} &  \num{0.149} & \num{0.73} \\
    75  & \num{1.177} &  \num{0.406} & \num{0.49} \\
    90  & \num{1.216} &  \num{0.948} & \num{0.12} \\
    105 & \num{1.486} &  \num{0.878} & \num{0.26} \\
    120 & \num{2.550} &  \num{0.480} & \num{0.68} \\
    135 & \num{2.500} &  \num{0.450} & \num{0.70} \\
    150 & \num{2.580} &  \num{0.480} & \num{0.69} \\
    165 & \num{1.930} &  \num{1.650} & \num{0.08} \\
    180 & \num{1.100} &  \num{1.030} & \num{0.03} \\
    \bottomrule
  \end{tabular}
\end{table}


\subsection{Brechungsindex von Glas}

Zur Bestimmung des Brechungsindex von Glas werden die beiden Glasscheiben von \SI{-2}{\degree} bis \SI{9}{\degree} um die Ruhelage gedreht und die Anzahl Nulldurchgänge des Spannungssignals gezählt. Aus der Anzahl der Nulldurchgänge wird daraufhin anhand Gleichung \eqref{eqn:n_Glas} der Brechungsindex bestimmt. Dies wird zehn mal wiederholt. Für den Winkel der beiden Glasplättchen $\Theta$, den Rotationswinkel $\delta$, die Dicke $T$ der Glasplättchen sowie die Wellenlänge $\lambda_\text{HeNe}$ gilt

\begin{align*}
  \Theta &= \SI{10}{\degree} & \delta &= \SI{11}{\degree} \\
  T &= \SI{1}{\milli\meter}  & \lambda_\text{HeNe} &= \SI{623.99}{\nano\meter}
\end{align*}

In Tabelle \ref{tab:Glas} sind die Anzahl der Nulldurchgänge, sowie die daraus bestimmten Brechungsindices aufgetragen. Für den Mittelwert des Brechungsindex von Glas folgt
\begin{equation}
  \overline{n} = \SI{1.54(2)} \, .
  \label{res:n_Glas}
\end{equation}

\begin{table}[H]
  \centering
  \caption{Anzahl der Nulldurchgänge und daraus berechnete Brechungsindices der Glasplättchen.}
  \label{tab:Glas}
  \begin{tabular}{cc}
    \toprule
    \# Nulldurchgänge & Brechungsindex $n$ \\
    \midrule
    42 & \num{1.66} \\
    38 & \num{1.56} \\
    35 & \num{1.49} \\
    35 & \num{1.49} \\
    38 & \num{1.56} \\
    35 & \num{1.49} \\
    36 & \num{1.52} \\
    37 & \num{1.54} \\
    37 & \num{1.54} \\
    36 & \num{1.52} \\
    \bottomrule
  \end{tabular}
\end{table}

\subsection{Brechungsindex Druckzelle}

Zur Bestimmung des Brechungsindex von Luft in Abhängigkeit des Drucks werden in Abbildung \ref{fig:Druck} die gezählten Nulldurchgänge $M$ jeder Messung gegen den Druck $p$ aufgetragen. Aufgrund von Problemen bei der Versuchsdurchführung fehlen in den Messreihen einzelne Messwerte. Für jede Messreihe wird aus den Nulldurchgängen $M$ nach Formel \ref{eqn:Nulldurchgänge}
\begin{equation}
   n = \frac{\lambda}{L} M + 1
   \label{eqn:Nulldurchgänge}
\end{equation}
der Brechungsindex $n$ bestimmt und ebenfalls gegen den Druck in Abbildung \ref{fig:Druck} dargestellt. Für jede Messreihe wird außerdem eine lineare Ausgleichsrechnung der Form
\begin{equation}
  n(p) = m \cdot p + n_0
  \label{eqn:Fit}
\end{equation}
durchgeführt, die Parameter $m$ und $n_0$ dieser Ausgleichsrechnung sind in Tabelle \ref{tab:Druck} aufgelistet.
Weiterhin gilt nach dem Lorentz-Lorenz Gesetz für die Steigung $m$ der Funktion
\begin{equation*}
  m = \frac{3 A}{2 R T}
\end{equation*}
mit der Allgemeinen Gaskonstanten $R$, der molaren Refraktivität $A$ und der Temperatur $T$.
Um einen vergleichbaren Wert unter Normalbedingungen ($T_0 = \SI{15}{\degreeCelsius}$, $p=\SI{1}{\atm}$) zu erhalten, wird anhand der Parameter und Gleichung \eqref{eqn:Fit} der Brechungsindex bei für $p=\SI{1}{\atm}$ bestimmt und mit
\begin{equation*}
  m = m \cdot \frac{T}{T_0}
\end{equation*}
die Steigung angepasst. Die Raumtemperatur $T$ wurde am Ende der Messung an mehreren Stellen im Interferometer gemessen und der Mittelwert $T = \SI{295.3(7)}{\kelvin}\, \widehat{=}\, \SI{22.1(7)}{\degreeCelsius}$ bestimmt.

Die so berechneten Brechungsindices unter Normalbedingung sind in Tabelle \ref{tab:Druck} aufgetragen.
Für den Mittelwert dieser folgt
\begin{equation}
  \overline{n} = \SI{1.000275(2)}{} \,.
  \label{eqn:n_exp}
\end{equation}


\begin{table}[H]
  \centering
  \caption{Parameter der Ausgleichsrechnungen und die daraus relsultierenden Brechungsindices bei Normalbedingung. Die Steigung $m$ ist noch nicht, wie durch Gleichung \eqref{eqn:Fit} beschrieben, angepasst.}
  \label{tab:Druck}
  \begin{tabular}{ccc}
    \toprule
      Steigung $m$ & Achsenabschnit $n_0$ & Brechungsindex \\
      \midrule
      \SI{2.70(3)e-07}{} & \SI{1.000003(1)}{} & \SI{1.000283(3)}{} \\
      \SI{2.35(3)e-07}{} & \SI{1.000002(2)}{} & \SI{1.000246(3)}{} \\
      \SI{2.67(2)e-07}{} & \SI{1.000002(1)}{} & \SI{1.000279(2)}{} \\
      \SI{2.76(3)e-07}{} & \SI{1.000002(1)}{} & \SI{1.000288(3)}{} \\
      \bottomrule
  \end{tabular}
\end{table}

\begin{figure}[H]
  \centering
  \begin{subfigure}[l]{.49\textwidth}
    \includegraphics[width = \textwidth]{Auswertung/Plots/Messwerte.png}
    \label{fig:Druck_M}
  \end{subfigure}
  \begin{subfigure}[r]{.49\textwidth}
    \includegraphics[width = \textwidth]{Auswertung/Plots/Brechungsindex.png}
    \label{fig:Druck_n}
  \end{subfigure}
  \caption{Brechungsindex von Luft als Funktion des Luftdrucks für vier Messreihen. Für die Messreihen 1. und 4. konnten für höhere Drücke keine Nulldurchgänge und somit auch kein Brechungsindex bestimmt werden. Alle aufgetragenen Werte sind in Tabelle \ref{tab:Messwerte} aufgelistet.}
  \label{fig:Druck}
\end{figure}

\begin{landscape}
  \begin{table}
    \small
    \caption{Messwerte für die Nulldurchgänge $M$ und die daraus berechneten Brechungsindices $n$. Die Werte sind ebenfalls in Abbildung \ref{fig:Druck} zu sehen.}
    \label{tab:Messwerte}
  \begin{tabular}{ccccccccc}
  \toprule
  Druck / $\si{\milli\bar}$ & \multicolumn{2}{l}{Messreihe 1} & \multicolumn{2}{l}{Messreihe 2} & \multicolumn{2}{l}{Messreihe 3} & \multicolumn{2}{l}{Messreihe 4} \\
   & $M_\text{1}$ & $n_\text{1}$ & $M_\text{2}$ & $n_\text{2}$ & $M_\text{3}$ & $n_\text{3}$ & $M_\text{4}$ & $n_\text{4}$ \\
  \midrule
  3,0   & 0   &                    1,0 \pm 0 &   0 &                    1,0 \pm 0 &   1 &  1,000006330 \pm 0,000000006 &   1  &  1,000006330 \pm 0,000000006 \\
  50,0  & 2   &  1,000012660 \pm 0,000000013 &   1 &  1,000006330 \pm 0,000000006 &   2 &  1,000012660 \pm 0,000000013 &   3  &  1,000018990 \pm 0,000000019 \\
  100,0 & 5   &  1,000031649 \pm 0,000000032 &   4 &  1,000025320 \pm 0,000000025 &   5 &  1,000031649 \pm 0,000000032 &   5  &  1,000031649 \pm 0,000000032 \\
  150,0 & 7   &  1,00004431 \pm 0,00000004 &   6 &  1,00003798 \pm 0,00000004 &   7 &    1,00004431 \pm 0,00000004 &   7  &    1,00004431 \pm 0,00000004 \\
  200,0 & 9   &  1,00005697 \pm 0,00000006 &   8 &  1,00005064 \pm 0,00000005 &   9 &    1,00005697 \pm 0,00000006 &   9  &    1,00005697 \pm 0,00000006 \\
  250,0 & 11  &  1,00006963 \pm 0,00000007 &  10 &    1,00006330 \pm 0,00000006 &  11 &    1,00006963 \pm 0,00000007 &  11  &    1,00006963 \pm 0,00000007 \\
  300,0 & 14  &  1,00008862 \pm 0,00000009 &  12 &    1,00007596 \pm 0,00000008 &  13 &    1,00008229 \pm 0,00000008 &  13  &    1,00008229 \pm 0,00000008 \\
  350,0 & 16  &  1,00010128 \pm 0,00000010 &  14 &    1,00008862 \pm 0,00000009 &  15 &    1,00009495 \pm 0,00000009 &  15  &    1,00009495 \pm 0,00000009 \\
  400,0 & 18  &  1,00011394 \pm 0,00000011 &  15 &    1,00009495 \pm 0,00000009 &  17 &    1,00010761 \pm 0,00000011 &  17  &    1,00010761 \pm 0,00000011 \\
  450,0 & 20  &  1,00012660 \pm 0,00000013 &  17 &    1,00010761 \pm 0,00000011 &  19 &    1,00012027 \pm 0,00000012 &  20  &    1,00012660 \pm 0,00000013 \\
  500,0 & 22  &  1,00013926 \pm 0,00000014 &  19 &    1,00012027 \pm 0,00000012 &  21 &    1,00013293 \pm 0,00000013 &  22  &    1,00013926 \pm 0,00000014 \\
  550,0 & 24  &  1,00015192 \pm 0,00000015 &  21 &    1,00013293 \pm 0,00000013 &  24 &    1,00015192 \pm 0,00000015 &  25  &    1,00015825 \pm 0,00000016 \\
  600,0 & 26  &  1,00016458 \pm 0,00000016 &  23 &    1,00014559 \pm 0,00000015 &  26 &    1,00016458 \pm 0,00000016 &  27  &    1,00017091 \pm 0,00000017 \\
  650,0 & 28  &  1,00017724 \pm 0,00000018 &  25 &    1,00015825 \pm 0,00000016 &  28 &    1,00017724 \pm 0,00000018 &  29  &    1,00018357 \pm 0,00000018 \\
  700,0 & 30  &  1,00018990 \pm 0,00000019 &  27 &    1,00017091 \pm 0,00000017 &  30 &    1,00018990 \pm 0,00000019 &  31  &    1,00019623 \pm 0,00000020 \\
  750,0 & 32  &  1,00020256 \pm 0,00000020 &  29 &    1,00018357 \pm 0,00000018 &  32 &    1,00020256 \pm 0,00000020 &  33  &    1,00020889 \pm 0,00000021 \\
  800,0 &     &                            &  31 &    1,00019623 \pm 0,00000020 &  33 &    1,00020889 \pm 0,00000021 &      &                            \\
  850,0 &     &                            &  32 &    1,00020256 \pm 0,00000020 &  36 &    1,00022788 \pm 0,00000023 &      &                            \\
  900,0 &     &                            &  33 &    1,00020889 \pm 0,00000021 &  39 &    1,00024687 \pm 0,00000025 &      &                            \\
  950,0 &     &                            &  35 &    1,00022155 \pm 0,00000022 &  41 &    1,00025953 \pm 0,00000026 &      &                            \\
  989,0 &     &                            &  36 &    1,00022788 \pm 0,00000023 &  42 &    1,00026586 \pm 0,00000027 &      &                            \\
  \bottomrule
  \end{tabular}
  \end{table}
\end{landscape}

%\newpage
%\section{Diskussion}
In Tabelle \ref{tab:ergebnisse} sind die experimentell bestimmten Werte für die Landé-Faktoren den theoretisch
berechneten Werten gegenübergestellt.
\begin{table}[H]
    \centering
    \caption{Vergeleich der experimentell ermittelten Landé-Faktoren mit den dazugehörigen Literaturwerten.}
    \label{tab:ergebnisse}
    \begin{tabular}{cc | c c | c}
      \toprule
      $\lambda \, / \, \si{\nano\meter}$ & Übergang & Landé-Faktor experimentell & Landé-Faktor theoretisch & Abweichung \\ 
      \midrule
        643,8 & $\sigma$ & 0,90 \pm 0,07 & 1    & 0,10 \\
        480,0 & $\pi$    & 0,43 \pm 0,04 & 0,5  & 0,14 \\
        480,0 & $\sigma$ & 1,80 \pm 0,16 & 1,75 & 0,03 \\
      \bottomrule
  \end{tabular}
 \end{table} \noindent
 Der größte vorliegende Fehler beträgt 14\% bei der Untersuchung der blauen Spektrallinie. Die
 durchgeführte Fehlerabschätzung schließt den Fehler nicht vollständig mit ein, sodass der Theoriewert 
 nicht im Bereich des experimentell bestimmten Wert inklusive Fehler liegt. 
 \\
 Eine der Fehlerquellen liegt in der Abschätzung der Abstände der Maxima. Aufgrund der geringen Helligkeit 
 der aufgenommenen Bilder, sind diese zunächst mit einem Bildbearbeitungsprogramm bearbeitet worden bevor sie
 ausgewertet worden sind. Trotzdem -oder durch die Bearbeitung bedingt- lag eine Unschärfe der Bilder vor, sodass
 das Abschätzen der Abstände zu Ungenauigkeiten geführt hat. \\
 \\
 Zudem ist eine genaue Einstellung des Magnetfeldes aufgrund der Skalierung des Stromgerätes nicht ideal möglich 
 gewesen. Dazu kommt, dass die Eichung des Magnetfeldes per Hand vorgenommen worden ist. Da 
 die Hallsonde sehr empfindlich bezüglich Winkeländerungen gegenüber des Magnetfelds ist, kann die
 Messung ungenau sein.\\
 Allgemein lässt sich jedoch sagen, dass die erzielten Ergebnisse als gut zu bewerten sind. Und somit wird im 
 Rahmen des durchgeführten Experiments die Theorie bestätigt. 
 

%\newpage
%\printbibliography
\end{document}
