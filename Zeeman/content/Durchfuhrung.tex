\section{Durchführung}
\subsection{Eichung des Magnetfeldes}
Während des Versuchs wird die Magnetfeldstärke nicht gemessen, sondern über den Spulenstrom eingestellt.
Es muss also zunächst die Abhängigkeit von Magnetfeldstärke zu Spulenstrom gemessen werden.
Dafür wird in einem Bereich von \SI{0}{\ampere} bis \SI{22}{\ampere} in Abständen von \SI{1}{\ampere} die Magnetfeldstärke mit einer Hallsonde bestimmt.

\subsection{Messung der roten Spetrallinie}

Die Übergänge, die zur Entstehung der roten Spektrallinie von Cadmium führen, entsprechen dem Übergang von $^1D_2$ zu $^1P_1$.
Für diese beiden Zustände gilt $S=0$ und damit $g_J = 1$, es handelt sich hierbei also um den normalen Zeeman-Effekt.
Das Magnetfeld wird nach Gleichung \eqref{eqn:B} auf eine Stärke von \SI{650}{\milli\tesla} eingestellt.
Sowohl die $\pi$- (\SI{0}{\degree}) also auch die $\sigma$-Linie (\SI{90}{\degree}) werden durch den Polarisationsfilter ausgewählt.
Durch den verschiebbaren Spalt des Versuchsaufbaus kann die rote Spetrallinie ausgewählt werden, die Digitalkamera wird so positioniert, dass ein deutliches Interferenzmuster aufgenommen werden kann.


\subsection{Messung der blauen Spektrallinie}
Der Übergang von $^3P_1$ zu $^3S_1$ erzeugt die blaue Spektrallinie von Cadmium.
Für beide Niveaus gilt $S=1$, somit folgt für die Landé-Faktoren $g_J(^3P_1) = \num{1.5}$ und $g_J(^3S_1) = \num{2}$.
Nach Gleichung \ref{eqn:an_zeeman} kann dann für die möglichen Übergange die Energie $\Delta E$ bestimmt werden.
Für Übergange mit $\Delta m = \pm 1$ unterscheiden sich die Energien für die unterschiedliche $m_1$ und $m_2$, weswegen der Landé-Faktor zu $|g_J| = \num{1.75}$
für die $\sigma$-Linie berechnet wird.
Für Übergange mit $\Delta m = 0$ beträgt der Landé-Faktor für alle möglichen Übergange $|g_J| = \num{0.5}$.
Dieser wird auch für die Einstellung des Magnetfeldes zur Messung der $\pi$-Linie genutzt.
Die zwei Magnetfelder der Stärke \SI{280}{\milli\tesla} für die $\pi$-Linie und \SI{1120}{\milli\tesla} für die $\sigma$-Linie werden daraufhin festgelegt.
Auch hier wird der Polarisationsfilter für die beiden Linien eingestellt.
