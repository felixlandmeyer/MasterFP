\section{Theoretische Grundlagen}
Ziel dieses Versuches ist die genauere Untersuchung des normalen und annormalen Zeeman-Effektes. 
Aus der resultierenden Aufspaltung der Spektrallinien des von einer Cadmium-Lampe ausgestrahltem 
unter dem Einfluss eines Magnetfeldes ist die Berechnung der Landé-Faktoren möglich. \\
\\
In Allgemeinen weisen Atome diskrete Energieniveaus auf. Befinden sie sich in einem äußeren homogenen 
Magnetfeld kommt es zu einer Aufspaltung dieser Energieniveaus. Dieser Effekt wird als Zeeman- Effekt
bezeichnet. Dieses Phänomen kann experimentell durch die Verschiebung der emittierten Spektrallinien 
untersucht werden. Mithilfe der sogenannten Auswahlregeln können vorab Aussagen über die 
Energie getroffen werden. \\

