\section{Einleitung}
Ziel dieses Versuches ist die genauere Untersuchung des normalen und annormalen Zeeman-Effektes.
Aus der resultierenden Aufspaltung der Spektrallinien des von einer Cadmium-Lampe ausgestrahltem
unter dem Einfluss eines Magnetfeldes ist die Berechnung der Landé-Faktoren möglich. \\

\section{Theoretische Grundlagen}
In Allgemeinen weisen Atome diskrete Energieniveaus auf. Befinden sie sich in einem äußeren homogenen
Magnetfeld, kommt es zu einer Aufspaltung dieser Energieniveaus. Dieser Effekt wird als Zeeman-Effekt
bezeichnet. Dieses Phänomen kann experimentell durch die Verschiebung der emittierten Spektrallinien
untersucht werden. Mithilfe der sogenannten Auswahlregeln können vorab Aussagen über die
Energie getroffen werden.

\subsection{Zeeman-Effekt}
Der Gesamtdrehimpuls $\vec{J}$ der Elektronen eines Atoms setzt sich aus dem Spin $\vec{S}$ und dem Bahndrehimpuls $\vec{L}$ der Elektronen zusammen.
Analog gilt für das magnetische Moment
\begin{equation*}
  \mu_J = \mu_L + \mu_S \text{mit} |\mu_J| = g_J \mu_B \sqrt{J (J + 1)} \, .
\end{equation*}
Hierbei beschreiben $\mu_B$ das Bohrsche Magneton und
\begin{equation}\label{eqn:Lande}
  g_J = \frac{3J(J + 1) + S(S + 1) − L(L + 1)}{2J (J + 1)}
\end{equation}
den Landé-Faktor des Drehimpulses.

Weiterhin kommt es zur Richtungsquantelung
\begin{equation*}
  \mu_{J,z} = −m g_J \mu_B, m \in \{ -J,\, -J+1, \, ... \, ,\, J-1, J \}
\end{equation*}
des magnetischen Moments in einem Magnetfeldes mit der Orientierungsquantenzahl $m$.
Die Anzahl an Einstellmöglichkeiten für $m$ beträgt somit $2J+1$.
Für die Energie $E$, die ein magnetischen Moment $\vec{\mu_J}$ in einem Magnetfeld $\vec{B}$ einnimmt, folgt demnach
\begin{equation*}
  E = -\vec{\mu_J} \vec{B} = m g_J \mu_B B \, .
\end{equation*}
Aufgrund der Energiequantelung durch $m$ entstehen so $2J+1$ zusätzliche Energieniveaus und somit auch Spektrallinien im Magnetfeld.
Diese Aufspaltung der Energieniveaus im Magnetfeld wird als Zeeman-Effekt bezeichnet.

Für $S=0$ ist $J=L$ und nach Gleichung \ref{eqn:Lande} immer $g_J = 1$.
Dies wird als \textbf{normaler Zeeman-Effekt} bezeichnet.
Für die Energiedifferenz $\Delta E$ zwischen zwei Zeeman-Aufspaltungen $m_1$ und $m_2$ folgt damit
\begin{equation*}
  \Delta E = \Delta m \mu_B B
\end{equation*}
mit $\Delta m  = m_2 - m_1$.
Die Auswahlregeln für Übergänge zwischen den Niveaus beziehen sich auf die Orientierungsquantenzahl.
So sind nur Übergänge erlaubt, für die $\Delta m = 0, \pm 1$ gilt.
Für $\Delta m = 0$ ist das senkrecht zum Magnetfeld ausgestrahlte Licht linear und parallel zum Magnetfeld polarisiert.
Diese Spektrallinien werden als $\pi$-Linien bezeichnet.
Bei $\Delta = \pm1$ sind die, senkrecht ausgestrahlten, Spektrallinien ebenfalls linear, aber senkrecht zum Magnetfeld, polarisiert.
Diese Linien werden als $\sigma$-Linien bezeichnet.

\par\bigskip

Für den Fall, dass $S \neq 0$ ist, wird die Aufspaltung der Energieniveaus als anomaler Zeeman-Effekt bezeichnet.
Die Auswahlregeln sind die selben, wie für den normalen Zeeman-Effekt, jedoch sind die Landé-Faktoren der Zustände unterschiedlich.
Somit folgt für die Energiedifferenz eines Übergangs von einem Niveau 1 in ein Niveau 2
\begin{equation} \label{eqn:an_zeeman}
  \Delta E = \left( m_1 g(J_1,L_1,S_1) - m_2 g(J_2,L_2,S_2) \right) \mu_B B = g_J \mu_B B \, .
\end{equation}
Hierbei hängt der Landé-Faktor von dem Übergang $m_1$ auf $m_2$ ab.
