\section{Diskussion}
In Tabelle \ref{tab:ergebnisse} sind die experimentell bestimmten Werte für die Landé-Faktoren den theoretisch
berechneten Werten gegenübergestellt.
\begin{table}[H]
    \centering
    \caption{Vergeleich der experimentell ermittelten Landé-Faktoren mit den dazugehörigen Literaturwerten.}
    \label{tab:ergebnisse}
    \begin{tabular}{cc | c c | c}
      \toprule
      $\lambda \, / \, \si{\nano\meter}$ & Übergang & Landé-Faktor experimentell & Landé-Faktor theoretisch & Abweichung \\ 
      \midrule
        643,8 & $\sigma$ & 0,90 \pm 0,07 & 1    & 0,10 \\
        480,0 & $\pi$    & 0,43 \pm 0,04 & 0,5  & 0,14 \\
        480,0 & $\sigma$ & 1,80 \pm 0,16 & 1,75 & 0,03 \\
      \bottomrule
  \end{tabular}
 \end{table} \noindent
 Der größte vorliegende Fehler beträgt 14\% bei der Untersuchung der blauen Spektrallinie. Die
 durchgeführte Fehlerabschätzung schließt den Fehler nicht vollständig mit ein, sodass der Theoriewert 
 nicht im Bereich des experimentell bestimmten Wert inklusive Fehler liegt. 
 \\
 Eine der Fehlerquellen liegt in der Abschätzung der Abstände der Maxima. Aufgrund der geringen Helligkeit 
 der aufgenommenen Bilder, sind diese zunächst mit einem Bildbearbeitungsprogramm bearbeitet worden bevor sie
 ausgewertet worden sind. Trotzdem -oder durch die Bearbeitung bedingt- lag eine Unschärfe der Bilder vor, sodass
 das Abschätzen der Abstände zu Ungenauigkeiten geführt hat. \\
 \\
 Zudem ist eine genaue Einstellung des Magnetfeldes aufgrund der Skalierung des Stromgerätes nicht ideal möglich 
 gewesen. Dazu kommt, dass die Eichung des Magnetfeldes per Hand vorgenommen worden ist. Da 
 die Hallsonde sehr empfindlich bezüglich Winkeländerungen gegenüber des Magnetfelds ist, kann die
 Messung ungenau sein.\\
 Allgemein lässt sich jedoch sagen, dass die erzielten Ergebnisse als gut zu bewerten sind. Und somit wird im 
 Rahmen des durchgeführten Experiments die Theorie bestätigt. 
 
