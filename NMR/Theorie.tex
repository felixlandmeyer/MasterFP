\section{Durchführung}
Der Versuch gliedert sich grob in zwei Abschnitte. Die Justage wird mit einem Gemisch aus Wasser und Kupfersulfat (\ce{CuSO_4}) und die Messungen der $T1$, $T2$ und der des Diffusionskoeffizienten an einer K1-Butanol Probe durchgeführt.

\subsection{Justage}
Zur Justage des Aufbaus wird die \ce{CuSO_4}-Probe in den Probenkopf gesteckt und die Startparameter aus Tabelle \ref{tab:Startparameter} eingestellt.

\begin{table}[H]
  \caption{Startparameter der Konsole. Zur Justage werden jediglich die Frequenz $F$ und die Phase $\phi$ verändert. Die Pulslänge $A$, die Anzahl der B-Pulse $N$, und die Periodendauer $P$ werden unverändert gelassen.}
  \label{tab:Startparameter}
  \tabular{cc}
  \toprule
  Frequenz $F$ & \SI{21.7}{\mega\hertz} \\
  Pulslänge $A$ & \SI{2}{\micro\second} \\
  Anzahl B-Pulse $N$ & 0
  Periode $P$ & \SI{0.5}{\second} \\
  Shims & $x = \num{1.0}, y = \num{-5.0}, z = \num{3.7}, z^2 = \num{-2.4}$ \\
  \bottomrule
\end{table}
