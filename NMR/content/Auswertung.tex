\section{Auswertung}

Alle Ausgleichsrechnungen werden mit dem Paket \texttt{scipy.optimize.curve\_fit}  aus \texttt{Python 3.7.3} durchgeführt.
Für Rechnungen mit fehlerbehafteten Größen wird das Paket \texttt{uncertainties} aus \texttt{Python 3.7.3} verwendet.

Zunächst wird eine Temperaturmessung und Justage durchgeführt. Mithilfe eines digitalen Thermometers wird die Temperatur 
auf \SI{21.5}{\degreeCelsius} bestimmt. Nach dem Einstellen der in der Versuchsanleitung \cite{anleitung} aufgeführten Startwerte
ergeben sich nach Varrieren die folgenden Werte
\begin{align}
  \text{Frequenz}&=\SI{21.7}{\mega\hertz} \\
  \text{Pulslänge}&=\SI{2}{\micro\second} \\
  \text{Anzahl Pulse}&=0 \\
  \text{Periode}&=\SI{0.5}{\second} \\
  \text{Phase} &= \SI{30}{\degree} \\
    x&=–1.0,y=–5.0,z=3,7,z2=–2,4 .
\end{align}

