\section{Diskussion}
Die in diesem Versuch ermittelten Relaxationszeiten für die untersuchte 1-Butanol-Probe sind
\begin{align}
  T_1 &= \SI{1.35(005)}{\second} \\
  T_2 &= \SI{0.56(005)}{\second} .
\end{align}
Diese Werte sind in der gleichen Größenordnung wie die Literaturwerte. Die theoretischen Wert entsprechen
für die beiden Relaxationszeiten  $T_1 = \SI{2.66}{\second}$ \cite{T1} und $T_2 = \SI{3(4)}{\second}$. 
Somit weichen die experimentell bestimmten Werte um $\SI{49.2}{\percent}$ bzw. $\SI{1}{\percent}$ von den Literaturwerten ab.
Die experimentell bestimmten Werte weichen stark von den Literaturwerten ab, wobei auch die Literaturwerte schwanken, jedoch
allesamt höher.
Allgemein lässt sich sagen, dass die sowohl die longitudinale als auch die transversale Relaxationszeit stark temperaturabhängig
sind. Eine Temperaturmessung bei jeder durchgeführten Messung hätte gegebenfalls zu genaueren Ergebnissen geführt,
als eine einmalige Temperaturmessung nach der Durchführung aller Versuchsteile. 
Zudem war eine genaue Temperaturmessung innerhalb der Probenkammer nicht möglich, sodass die gemessene Raumtemperatur als
Temperatur angenommen worden ist. \\
\\
Im Allgemein liegt eine Fehlerquelle in der vorgenommenen Justage. Wie bei der Bestimmung der $T_1$-Zeit auffällt liegt 
das gewünschte Verhältnis zwischen $M_0$ und $M_1$ nicht vor, sondern weicht um $\SI{11}{\percent}$ ab. 
Es ist nicht gelungen den Imaginärteil des Signal vollständig auszulöschen, sodass lediglich der Realteil zurückbleibt.
Dies kann zu einer Verfälschung der Messergebnisse führen. \\
\\
Die Bestimmung der Diffusionskonstante ergibt
\begin{align}
  D = \SI{2.41(011)e-10}{\square\meter\per\second}.
\end{align}
Dieser weicht vom Literaturwert von $D_\text{lit} = \SI{2.41(011)e-10}{\square\meter\per\second}$ \cite{Diff} um
$\SI{1}{\percent}$ ab. Diese Abweichung ist im Rahmen der Versuchsdurchführung als gut zu bewerten.
Die bei der Bestimmung des Molekülradius erzielten Ergebnisse weichen lediglich um $\SI{1.7}{\percent}$ voneinander ab. 
Obwohl bei der Rechnung die Annahmen eines ideal kugelförmigen Moleküls und einer hexagonal dichtesten Kugelpackung 
angenommen worden sind, liegen die Ergebnisse dicht beieinander. 

