\section{Diskussion}
Die in diesem Versuch ermittelten Relaxationszeiten für die untersuchte 1-Butanol-Probe sind
\begin{align*}
  T_1 &= \SI{1.35(005)}{\second} \\
  T_2 &= \SI{0.56(005)}{\second} \, .
\end{align*}
Für die $T_2$-Zeit und die Diffusionskonstante von Butanol konnten keine Literaturwerte gefunden werden, deshalb werden diese Werte im Folgenden mit denen von Wasser bzw. Pentanol verglichen.

Der theoretische Wert der $T_1$-Zeit entspricht $T_1 = \SI{2.66}{\second}$ \cite{T1}, somit weicht der experimentell bestimmte Wert um $\SI{49.2}{\percent}$ von dem Literaturwert ab.
%Ein Grund hierfür könnte eine nicht exakte Einstellung der Pulslängen für den \SI{90}{\degree}- und den \SI{180}{\degree}-Puls sein. Aufgrund eines zu kleinen Umklappens erscheinen die Signale kleiner und es wird eine kürzere $T_1$-Zeit bestimmt.


Die $T_2$-Zeit von Wasser beträgt nach \cite{chang} $T_2 = \SI{1.53(9)}{\second}$ und ist damit in etwa drei mal so lang wie die experimentell bestimmte $T_2$-Zeit von Butanol.
Durch die höhere Anzahl an Protonen in Butanol ist dort auch die Spin-Spin-Wechselwirkung größer und es kommt zu einer kleineren $T_2$ Zeit.

Allgemein lässt sich sagen, dass die sowohl die longitudinale als auch die transversale Relaxationszeit stark temperaturabhängig
sind. Eine Temperaturmessung bei jeder durchgeführten Messung hätte gegebenfalls zu genaueren Ergebnissen geführt,
als eine einmalige Temperaturmessung nach der Durchführung aller Versuchsteile.
Zudem war eine genaue Temperaturmessung innerhalb der Probenkammer nicht möglich, sodass die gemessene Raumtemperatur als
Temperatur angenommen worden ist. \\
\\
Im Allgemein liegt eine Fehlerquelle in der vorgenommenen Justage. Wie bei der Bestimmung der $T_1$-Zeit auffällt liegt
das gewünschte Verhältnis zwischen $M_0$ und $M_1$ nicht vor, sondern weicht um $\SI{11}{\percent}$ ab.
Es ist nicht gelungen den Imaginärteil des Signal vollständig auszulöschen, sodass lediglich der Realteil zurückbleibt.
Dies kann zu einer Verfälschung der Messergebnisse führen. \\


Die Bestimmung der Diffusionskonstante ergibt
\begin{align*}
  D = \SI{2.88(26)e-10}{\square\meter\per\second}\,.
\end{align*}
Dieser weicht stark von dem Literaturwert von Wasser $D_\text{W} = \SI{2.57(2)e-9}{\square\meter\per\second}$ \cite{wang} ab.
Ein Vergleich mit dem, von der Struktur her ähnlichen, 1-Pentanol liegt allerdings näher, da Pentanol nur eine Kohlenstoffverbindung mehr als 1-Butanol aufweist.
Die Diffusionskonstante von 1-Pentanol beträgt \SI{2.86e-10}{\square\meter\per\second} bei \SI{25}{\degreeCelsius} \cite{Holz} und liegt damit in der gleichen Größenordung wie die ermittelte Diffusionskonstante von 1-Butanol.


Die bei der Bestimmung des Molekülradius erzielten Ergebnisse aus \eqref{eqn:radius1} und \eqref{eqn:radius2} weichen lediglich um $\SI{1.7}{\percent}$ voneinander ab.
Obwohl bei der Rechnung die Annahmen eines ideal kugelförmigen Moleküls und einer hexagonal dichtesten Kugelpackung
angenommen worden sind, liegen die Ergebnisse dicht beieinander.
