\section{Durchführung}
Der Versuch gliedert sich grob in zwei Abschnitte. Die Justage wird mit einem Gemisch aus Wasser und Kupfersulfat (\ce{CuSO_4}) und die Messungen der $T1$, $T2$ und der des Diffusionskoeffizienten an einer K1-Butanol Probe durchgeführt.

\subsection{Justage}
Zur Justage des Aufbaus wird die \ce{CuSO_4}-Probe in den Probenkopf gesteckt und die Startparameter aus Tabelle \ref{tab:Startparameter} eingestellt.
Die Frequenz wird so eingestellt, dass das Signal möglichst wenig Schwingungen aufweist und exponentiell gegen Null fällt.
Weiterhin wird die Phase so gewählt, dass möglichst das gesamte Signal in einem Kanal zu sehen ist.
Mit den Drehreglern werden die Gradienten (Shims) und somit die Feldhomogenität verändert. Die Einstellungen werden so gewählt, dass der FID nach einem Zeitraum von \SI{2}{\milli\second} noch mindestens ein Drittel seiner maximalen Intensität besitzt.
Die Pulslänge des \SI{90}{\degree}-Puls wird so eingestellt, dass die Amplitude des FID maximal ist. Für den \SI{180}{\degree}-Puls wird die doppelte Zeit des \SI{90}{\degree}-Puls gewählt.

\subsection{T1-Messung}






\begin{table}[H]
  \centering
  \caption{Startparameter der Konsole. Zur Justage werden jediglich die Frequenz $F$ und die Phase $\phi$ verändert. Die Pulslänge $A$, die Anzahl der B-Pulse $N$, und die Periodendauer $P$ werden unverändert gelassen.}
  \label{tab:Startparameter}
  \begin{tabular}{ccc}
    \toprule
    Frequenz $F$ & \SI{21.7}{\mega\hertz} \\
    Pulslänge $A$ & \SI{2}{\micro\second} \\
    Anzahl B-Pulse $N$ & 0 \\
    Periode $P$ & \SI{0.5}{\second} \\
    Shims & $x = \num{1.0}, \,  y = \num{-5.0}, \, z = \num{3.7}, \, z^2 = \num{-2.4}$ \\
    \bottomrule
  \end{tabular}
\end{table}
