\section{Theorie}

\section{Spinbewegung im Magnetfeld}

Betrachtet man einen Atomkern mit einem Kernspin $I>0$, so besitzt dieser ein magnetisches Moment
\begin{equation*}
  \vec{\mu} = \gamma \vec{I}
\end{equation*}
mit dem gyromagnetischen Verhältnis $\gamma$.
Für Protonen mit Spin $I = \sfrac{1}{2}$ gilt
\begin{equation*}
  \gamma = 2 \pi \cdot \SI{42.577}{\mega\hertz\per\tesla}\,. %\cite{https://docs.scipy.org/doc/scipy/reference/constants.html}
\end{equation*}

In einem Magnetfeld $\vec{B_0} = B_0 \vec{e_z}$ erfährt ein Proton wegen seines magnetischen Moments ein Drehmoment $\vec{T}$.
Aus
\begin{align*}
  \vec{T} &= \vec{\mu} \times \vec{B_0}  \\
  \vec{T} &= \frac{\symup{d}\vec{I}}{\symup{d}t}
\end{align*}
folgt eine Spinpräzission
\begin{equation}
  \vec{I} =
  \begin{pmatrix*}[c]
    I_{\text{x},0} \, cos(\gamma B_0 t + \phi_0) \\
    -I_{\text{y},0} \, sin(\gamma B_0 t + \phi_0) \\
    I_{\text{z},0}
\end{pmatrix*}
 \, .
\end{equation}

Der Spin des Protons präzidiert also um die Richtungsachse des B-Felds, die Anfangsphase $\phi_0$ folgt dabei einer Gleichverteilung.
Die Frequenz $\omega_0 = \gamma B_0$ wird auch als Lamorfrequenz bezeichnet.
Die Bewegung des Spins und somit auch des magnetischen Moments des Protons in x-y Richtung kann in einer Leiterschleife eine Spannung mit der Lamorfrequenz induziert werden.
Für ein Volumen mit vielen Protonen mitteln sich die Anfangsphasen jedoch heraus und es wird keine Spannung gemessen.


Für die Komponente des Spins entlang des B$_{0}$-Feld folgt für Spin-$\sfrac{1}{2}$ Teilchen
\begin{equation*}
  I_\text{z} = \hbar m \quad \text{mit} \quad m = \pm\frac{1}{2} \, .
\end{equation*}
Die Ausrichtung der Spins entlang des Magnetfelds ist also entweder parallel (up) oder antiparallel (down) und
folgt der Boltzmann Verteilung
\begin{equation*}
  \frac{N_\text{up}}{N_\text{down}} = e^{\frac{\Delta E}{k_\text{B} T}} > 1
\end{equation*}
für
\begin{equation*}
  E = - \vec{\mu} \cdot \vec{B_0} = \mp \frac{1}{2} \hbar \omega_0 \Rightarrow \Delta E = \hbar \omega_0 \, .
\end{equation*}

Es sind also immer mehr Spins in dem energetisch günstigeren Zustand up und es entsteht eine feste Magnetisierung entlang des externen Magnetfelds B$_0$ (Längsmagnetisierung).

\subsection{Quermagnetisierung}

Um nun in einer Spule eine Spannung zu induzieren wird die Langsmagnetisierung der Probe auf die x-y-Ebende gekippt und die sogenannte Quermagnetisierung erzeugt.
Dazu wird zuerst ein Koordinatensystem eingeführt, welches mit der Lamorfrequenz rotiert, in diesem Koordinatensystem steht der Vektor des magnetischen Moments still.
Um diesen nun auf die x-y-Ebene zu kippen wird ein transversales Magnetfeld B$_1$ senkrecht zum ersten Magnetfeld B$_0$ eingeschaltet. Dieses, im rotierenden Koordinatensystem stehende, Magnetfeld entspricht einer linear polarisierten elektrmagnetischen Welle mit der Lamorfrequenz im nicht rotierenden Koordinatensystem.
Für das Transversalfeld ergibt sich der Flipwinkel
\begin{equation*}
  \alpha = \gamma B_1 \Delta t \, ,
\end{equation*}
welcher dem Winkel entspricht, um den die Längsmagnetisierung aus der z-Achse gekippt wird, dabei bezeichnet $B_1$ die Magnetfeldstärke des Transversalfelds und $\Delta t$ die Zeit, die das Transversalfeld eingeschaltet ist.

Nach dem Puls rotiert der Vektor des magnetischen Moments in der x-y-Ebende weiter um die z-Achse.
Dies induziert in einer Spule eine Spannung mit der Lamorfrequenz, in typischen NMR Versuchen wird dazu die Spule genutzt, mit der auch das B$_1$-Feld geschaltet wird.

\subsection{Relaxation}
Durch Wechselwirkungsprozesse bleiben die Spins nicht in der zuvor beschriebenen Lage in der x-y-Ebene, sondern relaxieren wieder in ihr Ursprungsausrichtung entlang der z-Achse.
Dies erzeugt in dem Spule ein exponentiell abfallendes Spannungssignal, welches Free Induction Decay (FID) genannt wird.

\subsubsection{Spin-Gitter Wechselwirkung, T1 Zeit}

Durch Wechselwirkung der Protonen mit umliegenden Molekülen können jene, durch Abgabe von Energiequanten $\Delta E = \hbar \omega_0$, in den Grundzustand zurückfallen.
Dies ist ein statistischer Prozess, der stark von dem untersuchten Material abhängt.
Die Gesamtmagnetisierung in z-Richtung baut sich mit der Zeitkonstanten T1 mit
\begin{equation}
  M_\text{z} = M_\text{z0} (1-e^{\frac{-t}{T1}})
  \label{eqn:T1Zeit}
\end{equation}
auf. Dabei beschreibt $M_\text{z,0}$ die Gesamtmagnetisierung vor dem Schalten des B$_1$-Feldes.

\subsubsection{Spin-Spin Wechselwirkung, T2-Zeit}

Das Magnetische Feld wird durch die Anwesenheit von anderen Spins lokal gestört und es herrscht eine andere Lamorfrequenz.
Dadurch dephasieren die Spins und die Quermagnetisierung nimmt mit
\begin{equation}
  M_\text{x,y} = M_\text{x0,y0} \cdot e^{\frac{t}{-T2}}
  \label{eqn:T2Zeit}
\end{equation}
ab.

















\section{Durchführung}
Der Versuch gliedert sich grob in zwei Abschnitte. Die Justage wird mit einem Gemisch aus Wasser und Kupfersulfat (\ce{CuSO_4}) und die Messungen der $T1$, $T2$ und der des Diffusionskoeffizienten an einer K1-Butanol Probe durchgeführt.

\subsection{Justage}
Zur Justage des Aufbaus wird die \ce{CuSO_4}-Probe in den Probenkopf gesteckt und die Startparameter aus Tabelle \ref{tab:Startparameter} eingestellt.
Die Frequenz wird so eingestellt, dass das Signal möglichst wenig Schwingungen aufweist und exponentiell gegen Null fällt.
Weiterhin wird die Phase so gewählt, dass möglichst das gesamte Signal in einem Kanal zu sehen ist.
Mit den Drehreglern werden die Gradienten (Shims) und somit die Feldhomogenität verändert. Die Einstellungen werden so gewählt, dass der FID nach einem Zeitraum von \SI{2}{\milli\second} noch mindestens ein Drittel seiner maximalen Intensität besitzt.
Die Pulslänge des \SI{90}{\degree}-Puls wird so eingestellt, dass die Amplitude des FID maximal ist. Für den \SI{180}{\degree}-Puls wird die doppelte Zeit des \SI{90}{\degree}-Puls gewählt.

\begin{table}[H]
  \centering
  \caption{Startparameter der Konsole. Zur Justage werden jediglich die Frequenz $F$ und die Phase $\phi$ verändert. Die Pulslänge $A$, die Anzahl der \textbf{B}-Pulse $N$, und die Periodendauer $P$ werden unverändert gelassen.}
  \label{tab:Startparameter}
  \begin{tabular}{ccc}
    \toprule
    Frequenz $F$ & \SI{21.7}{\mega\hertz} \\
    Pulslänge $A$ & \SI{2}{\micro\second} \\
    Anzahl \textbf{B}-Pulse $N$ & 0 \\
    Periode $P$ & \SI{0.5}{\second} \\
    Shims & $x = \num{1.0}, \,  y = \num{-5.0}, \, z = \num{3.7}, \, z^2 = \num{-2.4}$ \\
    \bottomrule
  \end{tabular}
\end{table}


\subsection{T1-Messung von K1-Butanol}
Zur Messung der T1-Zeit wird die Inversion Recovery genutzt.
Dazu wird der erste Puls (\textbf{A}) als ein \SI{180}{\degree}-Puls und der zweite Puls (\textbf{B}) als \SI{90}{\degree} geschaltet.
Es wird nur ein \SI{90}{\degree}-Puls geschaltet.
Die Periode, also der Abstand zwischen zwei \SI{180}{\degree}-Pulsen wird zuerst auf $P=\SI{10}{\second}$ gestellt.
Der Pulsabstand \tau zwischen Puls \textbf{A} und Puls \textbf{B} wird dann in einem Bereich von $\tau \in [\SI{1}{\milli\second}, \SI{10}{\second}]$ weiter erhöht, bis die maximale Amplitude des FID auf der negativen Achse liegt.
Für jeden Pulsabstand wird die Amplitude des FID nach Puls \textbf{A} gemessen, dafür muss gegebenenfalls Messbereich des Oszilloskops vergrößert werden, um das Signal nach Puls \textbf{A} zu detektieren.
Bei Werten von $\tau > \SI{1}{\second}$ wird die Periode auf $ P =\SI{10}{\second} + \tau$ gestellt.
Im Bereich des Nulldurchgangs werden mehr Messwerte als an den Rändern des Messbereichs bestimmt, um eine größere Genauigkeit für die Ausgleichsrechnung zur Bestimmung der T1-Zeit zu erziehlen.



\subsection{T2-Messung von K1-Butanol}
Zur Messung der T2-Zeit wird die Meiboom-Gill-Pulssequenz benutzt.
Die Pulslängen für \textbf{A} und \textbf{B} werden getauscht, sodass Puls \textbf{A} nun als \SI{90}{\degree} und Puls \textbf{B} als \SI{180}{\degree}-Puls geschaltet wird.
Die Anzahl der \textbf{B}-Pulse wird auf $N=100$ gestellt, die Periode wird auf das dreifache der T1-Zeit von K1-Butanol eingestellt.
In diesem Fall wird für die Periode $P = \SI{3.6}{\second}$ geschätzt.
Mit dem Schalter \textbf{MG} wird die Phase von Puls \textbf{B} um \SI{90}{\degree} zu Puls \textbf{A} verschoben.
Der Pulsabstand $tau$ wird so gewählt, dass die Echoamplitude nach dem letzten \textbf{B}-Puls in etwa \sfrac{1}{3} des ersten Echosignals entspricht.
Es wird mit dem Oszilloskop ein Bild einer Periode mit den 100 Echosignalen aufgenommen und die Daten des Oszilloskops gespeichert.
Weiterhin wird ein Bild einer Messung ohne die Phasenverschiebung von \SI{90}{\degree} aufgenommen. Dazu wird der Schalter \textbf{MG} auf \textbf{off} gestellt.

\subsection{Diffusionsmessung}
Zur Auswahl einer möglichst kleinen Schichtdicke wird am z-Gradienten der höchste Wert ausgewählt und umgepolt.
Die weiteren Gradienten bleiben unverändert.
Puls \textbf{A} wird weiterhin als \SI{90}{\degree} und Puls \textbf{B} als \SI{180}{\degree} geschaltet.
Es wird pro Periode ein Puls \textbf{B} geschaltet.
Die Periode bleibt weiterhin bei $P = \SI{3.6}{\second}$.
Nun wird der Pulsabstand ab $\tau = \SI{100}{\micro\second}$ erhöht und jeweils die Maximalamplitude des Echosignals nach Puls B gemessen.
Der Pulsabstand wird so lange erhöht, bis das Echosignal nicht mehr vom Untergrundrauschen zu unterscheiden ist.
Für eine Messung mit einem kleinen Pulsabstand wird das gesamte Echosignal mit dem Oszilloskop aufgenommen und abgespeichert.
