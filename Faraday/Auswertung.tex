\section{Auswertung}
Alle Ausgleichsrechnungen werden mit dem Paket \texttt{scipy.optimize.curve\_fit}  aus \texttt{Python 3.7.3} durchgeführt.
%Die Unsicherheit auf die gemessenen Ereignisse $N$ beträgt $\sigma_\text{N} = \sqrt{N}\,$.
Für Rechnungen mit fehlerbehafteten Größen wird das Paket \texttt{uncertainties} aus \texttt{Python 3.7.3} verwendet.

\subsection{Messung des Magnetfeldes}
Zunächst werden die aufgenommen Daten zur Bestimmung des Magnetfeldes untersucht. Die gemessenen Werte
sind in Tabelle \ref{tab:MF_Messung} aufgeführt.
\begin{table}[H]
    \centering
    \caption{Aufgenommene Werte bei der Bestimmung des Magnetfeldes mithilfe einer Hallsonde. Dabei sind die Werte der Postion $x$ die der Hallsonde auf einer Schiebeskala und ihre absoluten Beträge nicht von Bedeutung.}
    \label{tab:MF_Messung}
    \begin{tabular}{cc}
      \toprule
      Position $x$/$\si{\milli\meter}$ & B-Feld / $\si{\milli\tesla}$  \\
      \midrule
      115 & 50  \\
      120 & 180 \\
      125 & 221 \\
      126 & 222 \\
      127 & 223 \\
      128 & 224 \\
      129 & 224 \\
      130 & 224 \\
      131 & 222 \\
      132 & 219 \\
      133 & 215 \\
      134 & 209 \\
      135 & 201 \\
      140 & 89  \\
      \bottomrule
  \end{tabular}
 \end{table} \noindent
 In Abbildung \ref{fig:MF_Messung} sind die Magnetfeldstärken gegen die Position $x$ aufgetragen. Dabei wurde die Positionsskala neu skaliert, um eine Position im Luftspalt anzugeben.
 \begin{figure}[H]
     \centering
     \includegraphics[width=0.8\textwidth]{Auswertung/Plots/Magnetfeldstärke.png}
     \caption{Darstellung der aufgenommenen Werte des Magnetfeldes aufgetragen gegen die Position x der Hallsonde.}
     \label{fig:MF_Messung}
 \end{figure} \noindent
 Die Position einer Probe wird auf $\SI{16}{\milli\meter}$ im Luftspalt bestimmt. Dabei ist jedoch zu bedenken,
 dass die Proben unterschiedliche Dicken aufweisen.
 Somit wird in den folgenden Teilen der Auswertung für das vorherrschende Magnetfeld eine Stärke von
 \begin{align*}
     B_\text{Probe} = \SI{220.5(11)}{\milli\tesla}
 \end{align*} \noindent
 angenommen.

 \subsection{Messung der hochreinen GaAs-Probe}
 Als Referenzprobe wird eine hochreine, undotierte GaAs-Probe untersucht.
 Mit einer bekannten Dicke der Probe von $\SI{5.11}{\milli\meter}$ kann zu jeder der untersuchten
 Wellenlängen der normierte Drehwinkel bestimmt werden. Diese sind zusammen mit den gemessenen Werten, sowie den
 entsprechenden Wellenlängen in Tabelle \ref{tab:hr_Messung} aufgeführt.
 \begin{table}[H]
    \centering
    \caption{Aufgenommene Werte bei der Untersuchung der hochreinen GaAs-Probe als Referenz. Außerdem sind die
    berechneten normierten Rotationswinkel aufgetragen.}
    \label{tab:hr_Messung}
    \begin{tabular}{cccc}
      \toprule
      $\lambda\, / \, \si{\micro\meter}$ & $\Phi_\text{rein,1} \, / \, \si{\degree}$ & $\Phi_\text{rein,2} \, / \, \si{\degree}$ & $\Phi_\text{norm. rein} \, / \, \si{\radian\per\milli\meter}$  \\
      \midrule
      \num{1.060} & \num{250.63} & \num{275.68} & \num{0.0428}  \\
      \num{1.290} & \num{252.47} & \num{270.60} & \num{0.0310}  \\
      \num{1.450} & \num{255.73} & \num{270.42} & \num{0.0251}  \\
      \num{1.720} & \num{257.42} & \num{267.40} & \num{0.0170}  \\
      \num{1.960} & \num{254.67} & \num{261.28} & \num{0.0113}  \\
      \num{2.156} & \num{254.78} & \num{260.00} & \num{0.0089}  \\
      \num{2.340} & \num{229.43} & \num{239.22} & \num{0.0167}  \\
      \num{2.510} & \num{211.03} & \num{215.72} & \num{0.0080}  \\
      \num{2.650} & \num{247.65} & \num{253.57} & \num{0.0101}  \\
      \bottomrule
    \end{tabular}
   \end{table} \noindent
   In Abbildung \ref{fig:hr_Messung} sind die normierten Rotationswinkel in Abhängigkeit von der quadrierten
   Wellenlänge dargestellt.
   \begin{figure}[H]
       \centering
       \includegraphics[width=0.8\textwidth]{Auswertung/Plots/Messwerte_rein.pdf}
       \caption{Darstellung der berechneten normierten Rotationswinkel in Abhängigkeit von der Wellenlänge
       zum Quadrat. Aufgeführt sind in dieser Abbildung die Ergebnisse der Untersuchung der hochreinen GaAs-Probe. }
       \label{fig:hr_Messung}
   \end{figure}

   \subsection{Untersuchung der dotierten GaAs-Proben}
   Analog zu der hochreinen Probe werden zwei dotierte GaAs-Proben untersucht und die normierten Rotationswinkel
   bestimmt. Die Messwerte, sowie die berechneten Werte sind in Tabelle \ref{tab:Probe1} aufgeführt.
   In Tabelle \ref{tab:Probe2} sind die Ergebnisse der zweiten untersuchten Probe dargestellt.
   Die Dicke von Probe 1 beträgt $\SI{1.36}{\milli\meter}$ und von Probe 2 $\SI{1.296}{\milli\meter}$.
   \begin{table}[H]
     \centering
     \caption{Aufgenommene Werte bei der Untersuchung der ersten dotierten GaAs-Probe. Außerdem sind die
     berechneten normierten Roationswinkel aufgetragen.}
     \label{tab:Probe1}
     \begin{tabular}{ccccc}
       \toprule
       $\lambda\, / \, \si{\micro\meter}$ & $\Phi_\text{1,1} \, / \, \si{\degree}$ & $\Phi_\text{1,2} \, / \, \si{\degree}$ & $\Phi_\text{norm. 1} \, / \, \si{\radian\per\milli\meter}$  \\
       \midrule
       \num{1.060} & \num{259.03} & \num{269.33} & \num{0.0661}  \\
       \num{1.290} & \num{259.35} & \num{266.40} & \num{0.0452}  \\
       \num{1.450} & \num{261.15} & \num{267.67} & \num{0.0418}  \\
       \num{1.720} & \num{259.43} & \num{265.83} & \num{0.0411}  \\
       \num{1.960} & \num{261.67} & \num{261.92} & \num{0.0016}  \\
       \num{2.156} & \num{253.78} & \num{261.00} & \num{0.0463}  \\
       \num{2.340} & \num{229.17} & \num{236.00} & \num{0.0438}  \\
       \num{2.510} & \num{209.18} & \num{227.58} & \num{0.1181}  \\
       \num{2.650} & \num{258.42} & \num{257.27} & \num{0.0074}  \\
       \bottomrule
     \end{tabular}
     \end{table} \noindent

     \begin{table}[H]
       \centering
       \caption{Aufgenommene Werte bei der Untersuchung der zweiten dotierten GaAs-Probe. Außerdem sind die
       berechneten normierten Roationswinkel aufgetragen.}
       \label{tab:Probe2}
       \begin{tabular}{ccccc}
         \toprule
         $\lambda\, / \, \si{\micro\meter}$ & $\Phi_\text{2,1} \, / \, \si{\degree}$ & $\Phi_\text{2,2} \, / \, \si{\degree}$ & $\Phi_\text{norm. 2} \, / \, \si{\radian\per\milli\meter}$  \\
         \midrule
         \num{1.060} & \num{257.55} & \num{267.50} & \num{0.0670} \\
         \num{1.290} & \num{256.33} & \num{265.73} & \num{0.0633} \\
         \num{1.450} & \num{257.82} & \num{262.13} & \num{0.0291} \\
         \num{1.720} & \num{255.83} & \num{265.25} & \num{0.0634} \\
         \num{1.960} & \num{251.80} & \num{262.42} & \num{0.0715} \\
         \num{2.156} & \num{249.72} & \num{261.17} & \num{0.0771} \\
         \num{2.340} & \num{223.67} & \num{240.00} & \num{0.1100} \\
         \num{2.510} & \num{212.38} & \num{226.68} & \num{0.0963} \\
         \num{2.650} & \num{245.62} & \num{259.52} & \num{0.0936} \\
         \bottomrule
       \end{tabular}
       \end{table} \noindent
Zusätzlich werden die zuvor berechneten Werte für die undotierte Probe von den Werten der entsprechenden Wellenlänge nach
\begin{equation*}
 \Delta \Phi_\text{norm. i} = \Phi_\text{norm. i} - \Phi_\text{norm. rein}
\end{equation*}
abgezogen, um lediglich die freien Elektronen zu untersuchen.

In Abbildung \ref{fig:beideProben} sind die berechneten normierten Rotationswinkel beider Proben in
Abhängigkeit der quadratischen Wellenlänge dargestellt.
\begin{figure}[H]
  \centering
  \includegraphics[width=0.8\textwidth]{Auswertung/Plots/Fit_eff_mass.pdf}
  \caption{Darstellung der berechneten normierten Rotationswinkel für freie Elektronen der beiden untersuchten
  dotierten GaAs-Proben in Abhängigkeit des Wellenlängenquadrats. Zudem sind die linearen
  Ausgleichgeraden abgebildet.}
  \label{fig:beideProben}
\end{figure} \noindent
Für beide Proben wird eine Ausgleichgerade der
Form
\begin{equation*}
  \Delta \Phi_\text{norm.} = A \lambda^2
\end{equation*} \noindent
an die Werte angepasst. Daraus ergeben sich für die erste Probe der Parameter
\begin{align*}
  A_\text{Probe1} &= \SI{6.5(23)e3}{\radian\per\cubic\milli\meter}
  %b_\text{Probe1} &= \SI{0.007(26)}{\radian\per\milli\meter}.
\end{align*} \noindent
Analog ergibt sich
\begin{align*}
  A_\text{Probe2} &= \SI{1.41(10)e4}{\radian\per\cubic\milli\meter}
   %b_\text{Probe2} &= \SI{0.003(10)}{\radian\per\milli\meter}
\end{align*} \noindent
für die zweite untersuchte dotierte Probe. Daraus lassen sich mithilfe von Formel \eqref{eqn:effmass}
die effektiven Massen, sowie ihr Verhältnis zur Elektronenmasse bestimmen. Somit ergibt sich mit $n_\text{\ce{GaAs}} = 3,3$ \cite{n_GaAs}
\begin{align*}
  m_\text{eff. Probe1}^* &= \SI{5.2(10)e-32}{\kilo\gram}\, , \\
  V_1 &= \frac{m_\text{eff. Probe1}^*}{m_\text{e}} = 0.057 \pm 0.011
\end{align*}
für Probe 1. Analog ergibt sich für Probe 2
\begin{align*}
  m_\text{eff. Probe2}^* &= \SI{5.39(19)e-32}{\kilo\gram} \, , \\
  V_2 &= \frac{m_\text{eff. Probe2}^*}{m_\text{e}} = 0.059 \pm 0.002 \,.
\end{align*} \noindent

Für die Mittelwerte der effektiven Masse und des Verhältnis zur Elektronenmasse folgt somit
\begin{align}
  \overline{m_\text{eff.}} &= \SI{5.3(5)e-32}{\kilo\gram} \nonumber \, \\
  \overline{V} &= \num{0.058(6)} \, .
  \label{eqn:V_eff}
\end{align}
