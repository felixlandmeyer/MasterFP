\section{Diskussion}

Mit dem hier genutzen Aufbau lässt sich sehr gut die Drehung der Polarisation von polarisierten Licht durch den Faraday-Effekt erkennen.
Durch die Verwendung zweier Photowiderstände in Kombination mit dem Differenzverstärker lassen sich deutliche, unverauschte Signale auf dem Oszilloskop sehen.
Jedoch gab es schon bei der Justage Probleme mit der Messung der Polarisation bei ausgeschaltetem Magnetfeld.
Beim Durchgehen des Winkelbereichs der eingestrahlten Polarisation kam es nicht, wie erwartet, bei jeweils \SI{90}{\degree} zu einem Minimum des Differenzsignals,
sondern nach unterschiedlich großen Intervallen nahe der \SI{90}{\degree}.
Da dies trotz mehrmaligem Neujustieren immernoch der Fall war, wurde der Versuch dennoch durchgeführt.
\par \medskip
Weiterhin wirkte es schon während der Versuchsdurchführung so, dass die Einstellung des Glan-Taylor Prisma als Polarisator nicht reproduzierbar waren.
Eine Überprüfung dieser Vermutung wurde jedoch nicht durchgeführt.
Da das Differenzsignal des Selektivverstärkers immernoch die Form einer Sinusschwingung hatte, musste die Minimalamplitude des Signals direkt am Oszilloskops abgelesen werden.
Dies machte es schwierig, ein klares Minimum des Signals zu erkennen. Ein Tiefpassfilter als Integrator könnte ein konstantes Sigal erzeugen, welches am Oszilloskop deutlicher zu erkennen ist.
\par \medskip
Trotz dieser messtechnischen Probleme war es möglich, Drehwinkel der Polarisation durch das \ce{GaAs} zu bestimmen, die dem erwarteten Verlauf folgen (vgl. Abb. \ref{tab:hr_Messung}).
Die Messung der dotierten Proben und insbesondere die Bestimmung des Drehwinkels $\Delta \Phi_\text{norm.i}$ für die freien Ladungsträger weist erhebliche Streuung auf.
Aus Abbildung \ref{fig:beideProben} ist ersichtlich, dass die Drehwinkel deutlich streuen und im Falle von $\Delta \Phi_\text{norm. 1}$ nur einen linearen Verlauf erahnen lassen.
\par\medskip
Die Bestimmung der effektiven Masse der Ladungsträger über die linearen Ausgleichsrechnungn für die Drehwinkel $\Delta \Phi_\text{norm. i}$ liefert jedoch eine effektive Masse die nahe an dem Literaturwert liegt.
In Tabelle \ref{tab:Abweichung_lit} ist die Abweichung des Verhältnis der gemessenen, effektiven Masse der freien Ladungsträger zur Masse des Elektrons zu einem Literaturwert angegeben.
\begin{table}[H]
  \centering
  \caption{Abweichung des in \eqref{eqn:V_eff} bestimmten Verhältnis der effektiven Masse zur Elektronenmasse zum Verhältnis aus \cite{eff_lit}.}
  \label{tab:Abweichung_lit}
  \begin{tabular}{ccc}
    \toprule
    $\overline{V}_\text{exp.}$ & $V_\text{Lit.}$ &  Abweichung / $\si{\percent}$ \\
    \midrule
    \num{0.058(6)} & 0.067 & \num{13(8)} \\
    \bottomrule
  \end{tabular}
\end{table}
