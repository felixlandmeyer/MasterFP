\section(Aufbau)
In Abbildung \ref{fig:schemAufbau} ist der schematische Versuchsaufbau dargestellt. Verwendet wird
eine Halogenlampe, deren Wellenlängenbereich größtenteils im infraroten Bereich liegt.
\begin{figure}[H]
    \centering
    \includegraphics[width=0.8\textwidth]{schemAufbau.pdf}
    \caption{Schematische Abbildung des verwendeten Versuchsaufbaus zur Bestimmung der effektiven
     Masse mithilfe des Faraday-Effekts \cite{anleitung}.}
    \label{fig:schemAufbau}
\end{figure} \noindent
Mithilfe einer 
Kondensorlinse wird das Licht der Halogenlampe parallelisiert bevor es den Lichtzerhacker passiert.
Daraufhin durchläft das Licht das erste Glan-Taylor-Prisma. Dieses sorgt dafür, dass auf die sich im
Elektromagneten befindliche Probe das notwendige linear polarisierte Licht trifft. 
Nachdem das Licht die Probe und die Elektromagneten durchdrungen hat, trifft es auf einen Interferenzfilter.
In diesem Versuch werden neun unterschiedliche Interferenzfilter im Bereich von $\SIrange{1.06}{2.65}{\micro\meter}$
verwendet. Durch das zweite verwendete Glan-Thompson-Prisma werden die Polarisationsanteile des 
eintreffenden Lichtes getrennt, sodass zwei Teilstrahlen entstehen. Die Intensitäten dieser werden
individuell mithilfe von Photowiderständen gemessen und in den Differenzverstärker eingespeist. Dieser 
berechnet wie der Name vermuten lässt, die Differenzen der beiden eintrefenden Signale. Dieses
Differenzsignal wird daraufhin durch den Selektivverstärker verstärkt und vom Oszilloskop visualisiert. 
Die Nutzung zweier Photowiderstände bietet eine hohe Genauigkeit. Da aufgrund der Differenzrechnung das
Finden des Minimums einfacher ist. Zusätzlich ist eine Messung mit dieser Methode nicht so anfällig
gegenüber Wellenlängen- oder Intensitätsschwankungen der Halogenlampe.
Wenn stattdessen nur eine Photodiode zum Einsatz kommt, muss ein Minimum im gemessenen Signal gefunden 
werden. Oftmals würden jedoch eine Vielzahl von Drehwinkeln einem solchen Kriterium entsprechen 
und die Aufnahme der Messwerte würde ungenauer. 