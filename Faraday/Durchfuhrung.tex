\section{Aufbau}
In Abbildung \ref{fig:schemAufbau} ist der schematische Versuchsaufbau dargestellt. Verwendet wird
eine Halogenlampe, deren Wellenlängenbereich größtenteils im infraroten Bereich liegt.
\begin{figure}[H]
    \centering
    \includegraphics[width=0.8\textwidth]{images/SchemAufbau.pdf}
    \caption{Schematische Abbildung des verwendeten Versuchsaufbaus zur Bestimmung der effektiven
     Masse mithilfe des Faraday-Effekts \cite{anleitung}.}
    \label{fig:schemAufbau}
\end{figure} \noindent
Mithilfe einer Kondensorlinse wird das Licht der Halogenlampe parallelisiert bevor es den Lichtzerhacker passiert.
Daraufhin durchläft das Licht das erste Glan-Taylor-Prisma. Dieses sorgt dafür, dass auf die sich im
Elektromagneten befindliche Probe das notwendige linear polarisierte Licht trifft.
Nachdem das Licht die Probe und die Elektromagneten durchdrungen hat, trifft es auf einen Interferenzfilter.
In diesem Versuch werden neun unterschiedliche Interferenzfilter in einem Bereich von $\SI{1.06}{\micro\meter}$ bis $\SI{2.65}{\micro\meter}$
verwendet. Durch das zweite verwendete Glan-Thompson-Prisma werden die Polarisationsanteile des
eintreffenden Lichtes getrennt, sodass zwei Teilstrahlen entstehen. Die Intensitäten dieser werden
individuell mithilfe von Photowiderständen gemessen und in den Differenzverstärker eingespeist. Dieser
berechnet die Differenzen der beiden eintreffenden Signale.
Dieses Differenzsignal wird daraufhin durch den Selektivverstärker verstärkt und vom Oszilloskop visualisiert.\\
Die Nutzung zweier Photowiderstände bietet eine hohe Genauigkeit, da aufgrund der Differenzrechnung die
Bestimmung des Minimums einfacher ist. Zusätzlich ist eine Messung mit dieser Methode nicht so anfällig
gegenüber Wellenlängen- oder Intensitätsschwankungen der Halogenlampe.
Wenn stattdessen nur eine Photodiode zum Einsatz kommt, muss ein Minimum im gemessenen Signal gefunden
werden. Oftmals würden jedoch eine Vielzahl von Drehwinkeln einem solchen Kriterium entsprechen
und die Aufnahme der Messwerte würde ungenauer.

\section{Durchführung}
Bevor mit der eigentlichen Messung begonnen werden kann, muss eine Justage der Versuchsapparatur durchgeführt
werden. Dafür wird überprüft, ob die durch das zweite Glan-Thompson-Prisma entstehenden Teilstrahlen auf
die lichtempfindlichen Flächen der Photowiderstände treffen. Die Gehäuse von den Photowiderständen werden
entfernt und die Position des Glan-Thompson-Prismas angepasst. Ist eine gute Position der beiden
Teilstrahlen gegeben, wird der Lichtzerhacker eingeschaltet und die Frequenz des Selektivverstärkers auf
die Frequenz des Lichtzerhackers eingestellt. Gewüscht ist ein maximales Signal, wenn das Signal eines
Photowiderstandes direkt auf den Selektivverstärker gegeben wird. Ist dies der Fall wird der
Gütefaktor auf 100 geregelt. Zur letzten Überprüfung der Justage werden eine Probe, sowie ein
Interferenzfilter eingesetzt und die Signale der Photowiderstände auf die beiden Eingänge des
Differenzverstärkers gelegt. Beim Durchgehen eines großen Winkelbereichs am Polarisator sollte in Abständen von ca.
$\SI{90}{\degree}$ periodisch ein minimales Signal zu beobachten sein. Ist dies nicht der Fall, sollte
die Justage erneut vorgenommen werden. Ein besonderes Augenmerk sollte dabei auf dem Konstrast liegen.
Idealerweise verschwindet einer der Teilstrahlen, während der Andere eine maximale Intensität aufweist. \\
\\
Nachdem die Justage erfolgreich abgeschlossen ist, wird mit der eigentlichen Messung begonnen. Zunächst
wird eine hochreine Gallium-Arsenid Probe in den Versuchsaufbau eingefügt. Mithilfe des Oszilloskops
wird der Polarisationswinkel gesucht, bei dem sich das minimale Signal einstellt. Dieser wird notiert,
daraufhin das Magnetfeld umgepolt und der Vorgang wiederholt. Auf diese Weise wird die Probe im Zusammenspiel
mit den neun verschiedenen Interferenzfiltern untersucht. \\
Analog dazu werden zwei n-dotierte Gallium-Arsenid Proben untersucht. Dabei handelt es sich bei Probe 1
um eine $\SI{1.36}{\milli\meter}$ dicke Probe mit einer Ladungsträgerdichte von
$\SI{1.2e18}{\per\cubic\centi\meter}$. Die zweite Probe ist $\SI{1.296}{\milli\meter}$ dick und weist eine
Ladungsträgerdichte von $\SI{2,8e18}{\per\cubic\centi\meter}$ auf. \\
Im letzten Schritt der Durchführung wird das vorherrschende Magnetfeld in der Nähe des, im Elektromagneten
vorhandenen, Luftspaltes mithilfe einer Hallsonde ausgemessen und die Werte notiert.
