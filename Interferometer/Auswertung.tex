\section{Auswertung}
Der genutze Laser hat eine Wellenlänge von \SI{623.99}{\nano\meter}.

\subsection{Kontrast des Interferometers}
Um den maximalen Kontrast des Interferometers zu bestimmen, wird für jeden Polarisationswinkel $\phi$  der Kontrast aus den Diodenspannungen der Intensitätsmessung mit Formel \ref{eqn:Kontrast} berechnet und in Tabelle \ref{tab:Kontrast} aufgelistet.
Weiterhin werden in Abbildung \ref{fig:Kontrast} die berechneten Kontraste gegen den Polarisationswinkel aufgetragen und eine Ausgleichsrechnung der Form

\begin{equation*}
  K = A \cdot |\sin(2 \phi)|
\end{equation*}

durchgeführt. Als maximaler Kontrast folgt
\begin{equation}
  K_\text{max} = A = \num{0.71(5)} \, .
  \label{eqn:Kontrast}
\end{equation}

Es ist deutlich zu sehen, dass der maximale Kontrast deutlich kleiner als eins ist. Für die Durchfuhrung des Versuches wurde der Polarisationswinkel auf \SI{45}{degree} gestellt.

\begin{figure}[H]
  \centering
  \includegraphics[width = .5\textwidth]{Auswertung/Plots/Kontrast.png}
  \caption{Kontrast des Interferometers in Abhängigkeit des Polarisationswinkels. }
  \label{fig:Kontrast}
\end{figure}

\begin{table}[H]
  \centering
  \caption{Spannungen $U_\text{min}$, $U_\text{max}$ und Kontrast $K$ des Interferometers in Abhängigkeit des Polarisationswinkels $\phi$. Für die Versuchsdurchführung wurde der Polarisationswinkel auf \SI{45}{\degree} gestellt.}
  \label{tab:Kontrast}
  \begin{tabular}{cccc}
    \toprule
    $\phi \, / \,  \si{\degree}$ & $U_\text{max} \, / \, \si{\volt}$ &  $U_\text{min} \, / \, \si{\volt}$ & Kontast \\
    \midrule
    0   & \num{1.060} &  \num{0.880} & \num{0.09} \\
    15  & \num{0.634} &  \num{0.479} & \num{0.14} \\
    30  & \num{0.560} &  \num{0.243} & \num{0.24} \\
    40  & \num{0.692} &  \num{0.090} & \num{0.76} \\
    45  & \num{0.855} &  \num{0.095} & \num{0.80} \\
    50  & \num{0.967} &  \num{0.136} & \num{0.70} \\
    60  & \num{0.972} &  \num{0.149} & \num{0.73} \\
    75  & \num{1.177} &  \num{0.406} & \num{0.49} \\
    90  & \num{1.216} &  \num{0.948} & \num{0.12} \\
    105 & \num{1.486} &  \num{0.878} & \num{0.26} \\
    120 & \num{2.550} &  \num{0.480} & \num{0.68} \\
    135 & \num{2.500} &  \num{0.450} & \num{0.70} \\
    150 & \num{2.580} &  \num{0.480} & \num{0.69} \\
    165 & \num{1.930} &  \num{1.650} & \num{0.08} \\
    180 & \num{1.100} &  \num{1.030} & \num{0.03} \\
    \bottomrule
  \end{tabular}
\end{table}


\subsection{Brechungsindex von Glas}

Zur Bestimmung des Brechungsindex von Glas werden die beiden Glasscheiben von \SI{-2}{\degree} bis \SI{9}{\degree} um die Ruhelage gedreht und die Anzahl Nulldurchgänge des Spannungssignals gezählt. Aus der Anzahl der Nulldurchgänge wird daraufhin anhand Gleichung \eqref{eqn:n_Glas} der Brechungsindex bestimmt. Dies wird zehn mal wiederholt. Für den Winkel der beiden Glasplättchen $\Theta$, den Rotationswinkel $\delta$, die Dicke $T$ der Glasplättchen sowie die Wellenlänge $\lambda_\text{HeNe}$ gilt

\begin{align*}
  \Theta &= \SI{10}{\degree} & \delta &= \SI{11}{\degree} \\
  T &= \SI{1}{\milli\meter}  & \lambda_\text{HeNe} &= \SI{623.99}{\nano\meter}
\end{align*}

In Tabelle \ref{tab:Glas} sind die Anzahl der Nulldurchgänge, sowie die daraus bestimmten Brechungsindices aufgetragen. Für den Mittelwert des Brechungsindex von Glas folgt
\begin{equation}
  \overline{n} = \SI{1.54(2)} \, .
  \label{res:n_Glas}
\end{equation}

\begin{table}[H]
  \centering
  \caption{Anzahl der Nulldurchgänge und daraus berechnete Brechungsindices der Glasplättchen.}
  \label{tab:Glas}
  \begin{tabular}{cc}
    \toprule
    \# Nulldurchgänge & Brechungsindex $n$ \\
    \midrule
    42 & \num{1.66} \\
    38 & \num{1.56} \\
    35 & \num{1.49} \\
    35 & \num{1.49} \\
    38 & \num{1.56} \\
    35 & \num{1.49} \\
    36 & \num{1.52} \\
    37 & \num{1.54} \\
    37 & \num{1.54} \\
    36 & \num{1.52} \\
    \bottomrule
  \end{tabular}
\end{table}

\subsection{Brechungsindex Druckzelle}

Zur Bestimmung des Brechungsindex von Luft in Abhängigkeit des Drucks werden in Abbildung \ref{fig:Druck} die gezählten Nulldurchgänge $M$ jeder Messung gegen den Druck $p$ aufgetragen. Aufgrund von Problemen bei der Versuchsdurchführung fehlen in den Messreihen einzelne Messwerte. Für jede Messreihe wird aus den Nulldurchgängen $M$ nach Formel \ref{eqn:Nulldurchgänge}
\begin{equation}
   n = \frac{\lambda}{L} M + 1
   \label{eqn:Nulldurchgänge}
\end{equation}
der Brechungsindex $n$ bestimmt und ebenfalls gegen den Druck in Abbildung \ref{fig:Druck} dargestellt. Für jede Messreihe wird außerdem eine lineare Ausgleichsrechnung der Form
\begin{equation}
  n(p) = m \cdot p + n_0
  \label{eqn:Fit}
\end{equation}
durchgeführt, die Parameter $m$ und $n_0$ dieser Ausgleichsrechnung sind in Tabelle \ref{tab:Druck} aufgelistet.
Weiterhin gilt nach dem Lorentz-Lorenz Gesetz für die Steigung $m$ der Funktion
\begin{equation*}
  m = \frac{3 A}{2 R T}
\end{equation*}
mit der Allgemeinen Gaskonstanten $R$, der molaren Refraktivität $A$ und der Temperatur $T$.
Um einen vergleichbaren Wert unter Normalbedingungen ($T_0 = \SI{15}{\degreeCelsius}$, $p=\SI{1}{\atm}$) zu erhalten, wird anhand der Parameter und Gleichung \eqref{eqn:Fit} der Brechungsindex bei für $p=\SI{1}{\atm}$ bestimmt und mit
\begin{equation*}
  m = m \cdot \frac{T}{T_0}
\end{equation*}
die Steigung angepasst. Die Raumtemperatur $T$ wurde am Ende der Messung an mehreren Stellen im Interferometer gemessen und der Mittelwert $T = \SI{295.3(7)}{\kelvin}\, \widehat{=}\, \SI{22.1(7)}{\degreeCelsius}$ bestimmt.

Die so berechneten Brechungsindices unter Normalbedingung sind in Tabelle \ref{tab:Druck} aufgetragen.
Für den Mittelwert dieser folgt
\begin{equation}
  \overline{n} = \SI{1.000275(2)}{} \,.
  \label{eqn:n_exp}
\end{equation}


\begin{table}[H]
  \centering
  \caption{Parameter der Ausgleichsrechnungen und die daraus relsultierenden Brechungsindices bei Normalbedingung. Die Steigung $m$ ist noch nicht, wie durch Gleichung \eqref{eqn:Fit} beschrieben, angepasst.}
  \label{tab:Druck}
  \begin{tabular}{ccc}
    \toprule
      Steigung $m$ & Achsenabschnit $n_0$ & Brechungsindex \\
      \midrule
      \SI{2.70(3)e-07}{} & \SI{1.000003(1)}{} & \SI{1.000283(3)}{} \\
      \SI{2.35(3)e-07}{} & \SI{1.000002(2)}{} & \SI{1.000246(3)}{} \\
      \SI{2.67(2)e-07}{} & \SI{1.000002(1)}{} & \SI{1.000279(2)}{} \\
      \SI{2.76(3)e-07}{} & \SI{1.000002(1)}{} & \SI{1.000288(3)}{} \\
      \bottomrule
  \end{tabular}
\end{table}

\begin{figure}[H]
  \centering
  \begin{subfigure}[l]{.49\textwidth}
    \includegraphics[width = \textwidth]{Auswertung/Plots/Messwerte.png}
    \label{fig:Druck_M}
  \end{subfigure}
  \begin{subfigure}[r]{.49\textwidth}
    \includegraphics[width = \textwidth]{Auswertung/Plots/Brechungsindex.png}
    \label{fig:Druck_n}
  \end{subfigure}
  \caption{Brechungsindex von Luft als Funktion des Luftdrucks für vier Messreihen. Für die Messreihen 1. und 4. konnten für höhere Drücke keine Nulldurchgänge und somit auch kein Brechungsindex bestimmt werden. Alle aufgetragenen Werte sind in Tabelle \ref{tab:Messwerte} aufgelistet.}
  \label{fig:Druck}
\end{figure}

\begin{landscape}
  \begin{table}
    \small
    \caption{Messwerte für die Nulldurchgänge $M$ und die daraus berechneten Brechungsindices $n$. Die Werte sind ebenfalls in Abbildung \ref{fig:Druck} zu sehen.}
    \label{tab:Messwerte}
  \begin{tabular}{ccccccccc}
  \toprule
  Druck / $\si{\milli\bar}$ & \multicolumn{2}{l}{Messreihe 1} & \multicolumn{2}{l}{Messreihe 2} & \multicolumn{2}{l}{Messreihe 3} & \multicolumn{2}{l}{Messreihe 4} \\
   & $M_\text{1}$ & $n_\text{1}$ & $M_\text{2}$ & $n_\text{2}$ & $M_\text{3}$ & $n_\text{3}$ & $M_\text{4}$ & $n_\text{4}$ \\
  \midrule
  3,0   & 0   &                    1,0 \pm 0 &   0 &                    1,0 \pm 0 &   1 &  1,000006330 \pm 0,000000006 &   1  &  1,000006330 \pm 0,000000006 \\
  50,0  & 2   &  1,000012660 \pm 0,000000013 &   1 &  1,000006330 \pm 0,000000006 &   2 &  1,000012660 \pm 0,000000013 &   3  &  1,000018990 \pm 0,000000019 \\
  100,0 & 5   &  1,000031649 \pm 0,000000032 &   4 &  1,000025320 \pm 0,000000025 &   5 &  1,000031649 \pm 0,000000032 &   5  &  1,000031649 \pm 0,000000032 \\
  150,0 & 7   &  1,00004431 \pm 0,00000004 &   6 &  1,00003798 \pm 0,00000004 &   7 &    1,00004431 \pm 0,00000004 &   7  &    1,00004431 \pm 0,00000004 \\
  200,0 & 9   &  1,00005697 \pm 0,00000006 &   8 &  1,00005064 \pm 0,00000005 &   9 &    1,00005697 \pm 0,00000006 &   9  &    1,00005697 \pm 0,00000006 \\
  250,0 & 11  &  1,00006963 \pm 0,00000007 &  10 &    1,00006330 \pm 0,00000006 &  11 &    1,00006963 \pm 0,00000007 &  11  &    1,00006963 \pm 0,00000007 \\
  300,0 & 14  &  1,00008862 \pm 0,00000009 &  12 &    1,00007596 \pm 0,00000008 &  13 &    1,00008229 \pm 0,00000008 &  13  &    1,00008229 \pm 0,00000008 \\
  350,0 & 16  &  1,00010128 \pm 0,00000010 &  14 &    1,00008862 \pm 0,00000009 &  15 &    1,00009495 \pm 0,00000009 &  15  &    1,00009495 \pm 0,00000009 \\
  400,0 & 18  &  1,00011394 \pm 0,00000011 &  15 &    1,00009495 \pm 0,00000009 &  17 &    1,00010761 \pm 0,00000011 &  17  &    1,00010761 \pm 0,00000011 \\
  450,0 & 20  &  1,00012660 \pm 0,00000013 &  17 &    1,00010761 \pm 0,00000011 &  19 &    1,00012027 \pm 0,00000012 &  20  &    1,00012660 \pm 0,00000013 \\
  500,0 & 22  &  1,00013926 \pm 0,00000014 &  19 &    1,00012027 \pm 0,00000012 &  21 &    1,00013293 \pm 0,00000013 &  22  &    1,00013926 \pm 0,00000014 \\
  550,0 & 24  &  1,00015192 \pm 0,00000015 &  21 &    1,00013293 \pm 0,00000013 &  24 &    1,00015192 \pm 0,00000015 &  25  &    1,00015825 \pm 0,00000016 \\
  600,0 & 26  &  1,00016458 \pm 0,00000016 &  23 &    1,00014559 \pm 0,00000015 &  26 &    1,00016458 \pm 0,00000016 &  27  &    1,00017091 \pm 0,00000017 \\
  650,0 & 28  &  1,00017724 \pm 0,00000018 &  25 &    1,00015825 \pm 0,00000016 &  28 &    1,00017724 \pm 0,00000018 &  29  &    1,00018357 \pm 0,00000018 \\
  700,0 & 30  &  1,00018990 \pm 0,00000019 &  27 &    1,00017091 \pm 0,00000017 &  30 &    1,00018990 \pm 0,00000019 &  31  &    1,00019623 \pm 0,00000020 \\
  750,0 & 32  &  1,00020256 \pm 0,00000020 &  29 &    1,00018357 \pm 0,00000018 &  32 &    1,00020256 \pm 0,00000020 &  33  &    1,00020889 \pm 0,00000021 \\
  800,0 &     &                            &  31 &    1,00019623 \pm 0,00000020 &  33 &    1,00020889 \pm 0,00000021 &      &                            \\
  850,0 &     &                            &  32 &    1,00020256 \pm 0,00000020 &  36 &    1,00022788 \pm 0,00000023 &      &                            \\
  900,0 &     &                            &  33 &    1,00020889 \pm 0,00000021 &  39 &    1,00024687 \pm 0,00000025 &      &                            \\
  950,0 &     &                            &  35 &    1,00022155 \pm 0,00000022 &  41 &    1,00025953 \pm 0,00000026 &      &                            \\
  989,0 &     &                            &  36 &    1,00022788 \pm 0,00000023 &  42 &    1,00026586 \pm 0,00000027 &      &                            \\
  \bottomrule
  \end{tabular}
  \end{table}
\end{landscape}
