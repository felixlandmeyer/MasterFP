\section{Durchführung}

\subsection{Justage}
Zur Justage werden die beiden Spiegel M1 und M2 so eingestellt, dass der Laserstrahl mittig auf den PBSC des Interferometers fällt.
Weiterhin werden die Spiegel M$_\text{A,B,C}$ so eingestellt, dass der Strahl sie alle in der Mitte trifft und auch bei einem Durchgang des Strahls durch das Interferometer wieder auf die Mitte des PBSC fällt.\par
Im Detektinspfad hinter dem PBSC wird ein \SI{45}{\degree} Polarisationsfilter aufgebaut, damit es zu Interferenz der überlagerten Strahlen kommen kann.
Die Spiegel werden daraufhin so eingestellt, dass das Interferenzmuster auf dem Schirm verschwindet.
Die Strahlengänge im Interferometer sind dann parallel zueinander ausgerichtet.
\par\smallskip
Durch einen Verschiebung des Spiegels M2 wird der auf den PBSC fallende Strahl aufgeteilt.
Die Positionierung des Spiegels M2 wird so gewählt, dass die beiden aufgeteilten Strahlen auf die beiden Glasplättchen trifft, welche zwischen dem PBSC und dem Spiegel M$_\text{C}$ aufgebaut werden.

\subsection{Kontrastmessung}
Die Kontrastmessung des Interferometers in Abhängigkeit der eingestrahlten Polarisationsrichtung wird durch das Bestimmen von $I_\text{min}$ und $I_\text{max}$ für jede Polarisationsrichtung realisiert.
Der erste Polarisationsfilter wird auf Winkel zwischen \SI{0}{\degree} und \SI{180}{\degree} eingestellt und für jeden dieser Winkel die Intensitäten durch Drehen der Glasplättchen gesucht.
Der vermessene Winkelbereich wird in \SI{10}{\degree} Schritten gemessen, für hohe Kontraste werden diese auf \SI{5}{\degree} verringert.
Der Kontrast wird direkt vor Ort berechnet und der erste Polarisationsfilter auf den Winkel des höchsten Kontrasts eingestellt.

\subsection{Brechungsindex von Glas}

Der zweite Polarisationsfilter wird abgebaut und durch den zweiten PBSC und die beiden Photodioden ersetzt.
Die Glasplättchen werden um einen Winkel von $\delta = \SI{11}{\degree}$ gedreht und dabei die Anzahl der Nulldulgänge der Differenzspannung der Photodioden an einem Oszilloskop abgelesen.
Dieser Vorgang wird zehn mal wiederholt.
Für die Auswertung der Daten muss berücksichtigt werden, dass die Gasplättchen aufbaubedingt jeweils um einen Winkel von $\Theta = \SI{10}{\degree}$ zur Strahlrichtung gedreht sind.

\subsection{Brechungsindex Luft}

Zur Messung des Brechungsindex von Luft in Abhängigkeit des Luftdrucks werden die Glasplättchen aus den Strahlen entfernt und die Druckkammer in einen der Strahlengänge platziert.\par
Diese wird mit der Vakuumpumpe evakuiert.
Die Vakuumpumpe wird durch ein Ventil von der Druckkammer abgetrennt und diese durch ein weiteres Ventil langsam geflutet.
Während der Flutung der Kammer werden die Nulldulgänge der Differenzspannung elektronisch gezählt und für alle \SI{50}{\milli\bar} notiert.
Dieser Vorgang wird vier mal wiederholt.
