\section{Diskussion}
Die Messung des Kontrast des Sagnac Interferometer folgt in etwa dem erwarteten Verlauf des Betrags einer Sinuskuve.
Der maximale Kontrat ist, wie erwartet, bei einer Polarisation von \SI{45}{\degree} gegeben. Jedoch liegt der Kontrast, welcher über eine Ausgleichskurve bestimmt wurde etwa um \num{0.1} kleiner, als die maximal gemessene Kontrast.
\par\medskip
In Tabelle \ref{tab:n_Abw} sind die Abweichungen der experimentellen Daten zu Literaturwerten aufgelistet.
Der Brechungsindex der Glasplättchen beträgt nach Herstellerangaben $n_\text{Lit.} = \num{1.5}$.
Der Brechungsindex von Luft beträgt nach \cite{n_Luft} unter Normalbedingungen $n_\text{Lit.} = \num{1.000277}$.
Für die Berechnung der Abweichung des Brechungsindex von Luft wird mit $n_\text{Lit.} \to n_\text{Lit.} - 1 $ und $n_\text{Exp.} \to n_\text{Exp.} -1$ gerechnet, da die Abweichung der signifikanten Stellen interessant ist.
Die Abweichungen von den Literaturwerten sind minimal, zumal die Vermutung naheliegt, dass die Herstellerangabe für Glas nicht auf die zweite Dezimalstelle genau ist.

\begin{table}[H]
  \centering
  \caption{Abweichung der experimentell bestimmten Brechungsindices zu Literaturwerten.}
  \label{tab:n_Abw}
  \begin{tabular}{cccc}
    \toprule
    Material &  $n_\text{Exp.}$ & $n_\text{Lit.}$ & Abweichung / $\si{\percent}$  \\
    \midrule
    Glas & \num{1.54(2)} & \num{1.5} & \num{2.4(10)} \\
    Luft & \num{1.000275(2)} & \num{1.000277} & \num{0.9(6)}\\
    \bottomrule
  \end{tabular}
\end{table}
