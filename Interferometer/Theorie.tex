\section{Theorie}
Ziel dieses Versuches ist es mithilfe eines Sagnac-Interferometers den Brechungsindex von Glas
zu bestimmen. Außerdem wird Brechungsindex von Luft in Abhängigket der vorherrschenden 
Drucks untersucht. \\
\\ 
Bei dem in diesem Versuch genutze Interferometer handelt es sich um ein Sagnac-Interferometer. 
Genauso wie bei anderen Interferometern nutzt es den Effekt der Interferenz, um beispielsweise 
Brechungsindices bestimmen zu können. Damit es zur Interferenz kommen kann, müssen zwei linear 
polarisierte Signale uberlägert werden. Zur Erzeugung der zwei notwendigen Signale wird im Sagnac-Interferometer 
ein Strahlteilerwürfel genutzt. Dieser setzt sich aus zwei Prismen zusammen. Durch die Grenzfläche 
zwischen diesen zwei Prismen kommt es teilweise zur Reflexion beziehungsweise zur Transmission. 
Das eintreffende kohärente, linear polarisierte Laserlicht wird dadurch in seinen horizontal und 
vertikal polarisierten Anteil aufgespalten. \\
Im Unterschied zu einem Michelson-Interferometer legen die beiden Teilstrahlen den selben Weg
innerhalb des Interferometers zurück, lediglich in entgegengesetze Richtungen. Im
Michelson-Interferometer werden die Teilstrahlen durch einen halbdurchlässigen Spiegel erzeugt,
auf zwei separate Spiegeln reflektiert und anschschließend wieder zusammengeführt bevor die 
Interferenzerscheinung genauer untersucht werden kann.
\\
Um die Qualität des vom verwendeteten Interferometers erzeugten Interferenzbild beurteilen zu können,
werden die Intensitäten benachbarter Minima und Maxima verglichen. Dieser Zusammenhang wird 
als Kontrast bezeichnet und ist durch 
\begin{equation}
    K = \frac{I_\text{max}-I_\text{min}}{I_\text{max}+I_\text{min}}
\end{equation} \noindent 
gegeben. Dabei varrieren die Werte zwischen 0 und 1, wobei 1 den bestmöglichen Kontrast angibt. 